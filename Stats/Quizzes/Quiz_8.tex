\documentclass[addpoints, 12pt]{exam}

\setlength{\topmargin}{-.75in} \addtolength{\textheight}{2.00in}
\setlength{\oddsidemargin}{.00in} \addtolength{\textwidth}{.25in}

\usepackage{amsmath,color,graphicx, multicol}
\usepackage[inline]{asymptote}

\nofiles

\setlength{\parindent}{0in}

\pagestyle{plain}

\title{Quiz template}
\author{Wray}
\begin{document}

\begin{asydef}
//
// Global Asymptote definitions can be put here.
//
usepackage("bm");
texpreamble("\def\V#1{\bm{#1}}");
\end{asydef}



\noindent {\sc {\bf {\Large Quiz 8: Probability of unions and complements, conditional probability}}
            \hfill MAT 123, Summer 2016}
\bigskip

\noindent {\sc  {\large Name:}
             \hfill}
             
\bigskip
\bigskip

\smallskip

\begin{questions}

\question[10]
We conduct and experiment by flipping a coin and drawing a card from a deck.
\begin{parts}
\part
What is the probability that the coin shows tails?
\vspace{\stretch{2}}

\part
What is the probability that the card is a diamond?
\vspace{\stretch{2}}

\part
What is the probability of flipping tails and drawing a diamond?
\vspace{\stretch{2}}

\part
What of the probability of flipping tails or drawing a diamond?

\vspace{\stretch{3}}

\end{parts}

\newpage

\question[10]
An urn contains 10 red chips and 2 green chips.  Two chips are drawn from the jar.
\begin{parts}
\part
What is the probability that the first chip is green?

\vspace{\stretch{2}}

\part
If the first chip is green, what is the probability that the second chip is red?

\vspace{\stretch{2}}

\part
What is the probability of getting two red chips?

\vspace{\stretch{2}}

\part
What is the probability that the second chip is red?

\vspace{\stretch{3}}

\end{parts}

\end{questions}

\end{document}