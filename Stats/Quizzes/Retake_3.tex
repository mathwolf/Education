\documentclass[addpoints, 12pt]{exam}

\setlength{\topmargin}{-.75in} \addtolength{\textheight}{2.00in}
\setlength{\oddsidemargin}{.00in} \addtolength{\textwidth}{.25in}

\usepackage{amsmath,color,graphicx, multicol}
\usepackage[inline]{asymptote}

\nofiles

\setlength{\parindent}{0in}

\pagestyle{plain}

\title{Quiz template}
\author{Wray}
\begin{document}

\begin{asydef}
//
// Global Asymptote definitions can be put here.
//
usepackage("bm");
texpreamble("\def\V#1{\bm{#1}}");
\end{asydef}



\noindent {\sc {\bf {\Large Retake 3: Functions}}
            \hfill MAT 123, Summer 2016}
\bigskip

\noindent {\sc  {\large Name:}
             \hfill}
             
\bigskip
\bigskip

\smallskip

\begin{questions}

\question[5]
Let $f(x) = 2x^2 + x$.  Find $f(b-2)$.

\vspace{\stretch{3}}

\question[5]

Recall the shape of the basic graph $y = x^2$.  Use transformations to make a sketch of the related function $y = \dfrac{1}{2} (x+2)^2 - 1$.  

\vspace{\stretch{5}}

\newpage

\question[10]

Put the function
\begin{equation*}
	f(x) = -\dfrac{1}{3} x^2 + 2x - 17
\end{equation*}
into vertex form by completing the square.

\end{questions}

\end{document}