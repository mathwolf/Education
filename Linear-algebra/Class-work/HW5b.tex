\documentclass[addpoints, 12pt]{exam}

\setlength{\topmargin}{-.75in} \addtolength{\textheight}{2.00in}
\setlength{\oddsidemargin}{.00in} \addtolength{\textwidth}{.25in}

\usepackage{amsmath, amssymb, color,graphicx, multicol, gensymb, nicefrac}
\usepackage[inline]{asymptote}

\renewcommand{\vec}[1]{\mathbf{#1}}

\newcommand{\real}{\mathbb{R}}

\nofiles

\setlength{\parindent}{0in}

\pagestyle{plain}

\title{Linear algebra homework template}
\author{Wray}
\begin{document}

\begin{asydef}
//
// Global Asymptote definitions can be put here.
//
usepackage("bm");
texpreamble("\def\V#1{\bm{#1}}");
\end{asydef}



\noindent {\sc {\bf {\Large Homework 5b}}
            \hfill MAT 255, Fall 2016}
\bigskip
\bigskip
\smallskip

\begin{questions}

\question
Let $A$ be an $\ell \times m$ matrix, and let $B,C$ be $m \times n$ matrices.  Show that $A(B+C) = AB + AC$.

\question
Let $A$ be an $\ell \times m$ matrix, and let $B$ be an $m \times n$ matrix.  Show that $(AB)^T = B^T A^T$.

\question
Let $A,B,C$ be matrices.  Suppose that $AB = AC$ and that the matrix $A$ is invertible.  What assumptions are necessary about the sizes of the matrices in order for these statements to hold?  Show the steps we use to conclude that $B = C$.

\question
Let $A$ be a $2 \times 2$ matrix with entries
\begin{equation*}
\left[ \begin{array}{cc}
a & b \\
c & d
\end{array}
\right].
\end{equation*}
Prove that $A$ has an inverse if and only if $ad - bc \ne 0$.  You will use the following ideas in your proof.
\begin{parts}
\part
If $ad - bc = 0$, then the equation $A \vec{x} = \vec{0}$ has more than one solution.  First, consider the case when $a = b = 0$.  Next, look at the case where $a$ and $b$ are not both $0$.
\part
If $ad - bc \ne 0$, then the matrix
\begin{equation*} \dfrac{1}{ad-bc}
\left[ \begin{array}{cc}
d & -b \\
-c & a
\end{array}
\right]
\end{equation*}
satisfies the definition of an inverse.
\end{parts}

\question
Suppose $T: \real^n \to \real^n$ is a linear transformation with the property that there are distinct vectors $\vec{u}, \vec{v}$ in $\real^n$ such that $T(\vec{u}) = T(\vec{v})$.  Is $T$ an onto transformation?  Justify your answer.

\end{questions}

\end{document}