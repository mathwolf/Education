\documentclass[addpoints, 12pt]{exam}

\setlength{\topmargin}{-.75in} \addtolength{\textheight}{2.00in}
\setlength{\oddsidemargin}{.00in} \addtolength{\textwidth}{.25in}

\usepackage{amsmath,color,graphicx, multicol, amssymb}
\usepackage[inline]{asymptote}

\nofiles

\newcommand{\real}{\mathbf{R}}

\setlength{\parindent}{0in}

\pagestyle{plain}
\newcommand{\poly}{\mathbf{P}}

\title{Axioms for a vector space}
\author{Wray}
\begin{document}

\begin{asydef}
//
// Global Asymptote definitions can be put here.
//
usepackage("bm");
texpreamble("\def\V#1{\bm{#1}}");
\end{asydef}



\noindent {\sc {\bf {\Large Axioms for a vector space}}}
\bigskip
\bigskip

In the following, we assume that $\vec{u}, \vec{v}$, and $\vec{w}$ are any vectors in the vector space $V$ and that $c$ and $d$ are any scalars.   
\begin{questions}

\question
$\vec{u} + \vec{v} = \vec{v} + \vec{u}$

\question
$(\vec{u} + \vec{v}) + \vec{w} = \vec{u} + (\vec{v} + \vec{w})$

\question
There exists a vector $\vec{0}$ in $V$ such that
\begin{equation*}
	\vec{v} + \vec{0} = \vec{v}
\end{equation*}
for every vector $\vec{v}$.

\question
For every vector $\vec{u}$ in the space there is an additive inverse, denoted $-\vec{u}$, such that
\begin{equation*}
	\vec{u} + -\vec{u} = \vec{0}.
\end{equation*}

\question
$c(\vec{u} + \vec{v}) = c \vec{u} + c \vec{v}$

\question
$(c + d) \vec{u} = c \vec{u} + d \vec{u}$

\question
$c (d \vec{u}) = (cd) \vec{u}$

\question
$1 \vec{u} = \vec{u}$

\end{questions}

\end{document}