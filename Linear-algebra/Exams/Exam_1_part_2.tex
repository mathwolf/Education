\documentclass[addpoints, 12pt]{exam}

\setlength{\topmargin}{-.75in} \addtolength{\textheight}{2.00in}
\setlength{\oddsidemargin}{.00in} \addtolength{\textwidth}{.25in}

\usepackage[inline]{asymptote}
\usepackage{amsmath, gensymb, nicefrac, verbatim, graphicx, multicol, arydshln}
\usepackage[export]{adjustbox}

\nofiles

\pagestyle{empty}

\setlength{\parindent}{0in}
\newcommand{\real}{\mathbf{R}}

\title{Linear Exam 1 part 2 - Fall 2016}
\author{Wray}
\begin{document}

\begin{asydef}
//
// Global Asymptote definitions can be put here.
//
usepackage("bm");
texpreamble("\def\V#1{\bm{#1}}");
\end{asydef}

\noindent {\sc {\bf {\Large Exam 1, part 2}}
            \hfill MAT 255, Fall 2016}
\bigskip

\noindent {\sc  {\large Name:}
             \hfill}
             
\bigskip
You can use a calculator on the following problems.

\begin{questions}
\setcounter{question}{2}
\question[10]
\begin{parts}
\part
Applying Gaussian elimination to the linear system
\begin{gather*}
	3x_1 + 4x_2 + x_3 = 9 \\
    6x_1 + 5x_2 + 2x_3 + x_4 = 9
\end{gather*}
produces the reduced row echelon form
\begin{gather*}
	\left[ \begin{array}{cccc:c}
	1 & 0 & \nicefrac{1}{3} & \nicefrac{4}{9} & -1 \\
    0 & 1 & 0 & - \nicefrac{1}{3} & 3
	\end{array} \right].
\end{gather*}
What are the solutions to the linear system? Use parameters for any free variables.
\vspace{\stretch{2}}

\part
Applying Gaussian elimination to the linear system
\begin{gather*}
	x_1 + 4x_2 + 3x_3 + 4x_4 = 16 \\
    2x_1 + 8x_2 + 6x_3 + 8x_4 = 8
\end{gather*}
produces the reduced row echelon form
\begin{gather*}
	\left[ \begin{array}{cccc:c}
	1 & 4 & 3 & 4 & 0 \\
    0 & 0 & 0 & 0 & 1
	\end{array} \right].
\end{gather*}
What are the solutions to the linear system? Use parameters for any free variables.
\vspace{\stretch{3}}
\end{parts}

\clearpage
\question[20]
\begin{parts}
\part
Describe the shape made by the span of the vectors
$\left[ \begin{array}{c}
1 \\ 1 \\ 0
\end{array} \right]$ and
$\left[ \begin{array}{c}
1 \\ 0 \\ -1
\end{array} \right]$.
\vspace{\stretch{1}}

\part
Does the vector
$\vec{v} = \left[ \begin{array}{c}
4 \\ 2 \\ -2
\end{array} \right]$
lie in
$\textnormal{Span} \left\lbrace \left[ \begin{array}{c}
1 \\ 1 \\ 0
\end{array} \right], 
\left[ \begin{array}{c}
1 \\ 0 \\ -1
\end{array} \right]
\right\rbrace$?
\vspace{\stretch{3}}

\part
For what values of the parameter $k$ does the vector 
$\vec{v} = \left[ \begin{array}{c}
0 \\ 4 \\ k
\end{array} \right]$
lie in
$\textnormal{Span} \left\lbrace \left[ \begin{array}{c}
1 \\ 1 \\ 0
\end{array} \right], 
\left[ \begin{array}{c}
1 \\ 0 \\ -1
\end{array} \right]
\right\rbrace$?
\vspace{\stretch{4}}

\end{parts}

\clearpage
\question[18]
Determine if the given vectors form a linearly independent set.  Explain the reasons that support your conclusion.
\smallskip
\begin{parts}
\part
$\vec{u} = \left[
\begin{array}{c}
	1 \\ 0 \\ 1
\end{array}
\right],
\vec{v} = \left[
\begin{array}{c}
	2 \\ 3 \\ 4
\end{array}
\right],
\vec{w} = \left[
\begin{array}{c}
	2 \\ 6 \\ 6
\end{array}
\right]$
\vspace{\stretch{2}}

\part
$\vec{u} = \left[
\begin{array}{c}
	1 \\ 0 \\ 1
\end{array}
\right],
\vec{v} = \left[
\begin{array}{c}
	2 \\ 3 \\ 4
\end{array}
\right],
\vec{w} = \left[
\begin{array}{c}
	2 \\ 6 \\ 8
\end{array}
\right]$
\vspace{\stretch{2}}

\part
$\vec{u} = \left[
\begin{array}{c}
	1 \\ 0 \\ 1
\end{array}
\right]$
\vspace{\stretch{2}}
\end{parts}

\clearpage
\question[15]
Let $T: \real^2 \to \real^2$ be the following linear transformation.  First the plane is rotated $90 \degree$ in a counterclockwise (positive) direction.  Next, the plane is reflected across the y-axis.  Finally, the plane is rotated $45 \degree$ in a clockwise (negative) direction. 
\begin{parts}
\part
Draw pictures that show the locations of the unit coordinate vectors $\vec{e_1}$ and $\vec{e_2}$ before and after the transformation $T$.
\vspace{\stretch{3}}

\part
Write the matrix for this transformation. 
\vspace{\stretch{2}}

\part
Find the image of the vector $\vec{v} = \left[
\begin{array}{c}
1 \\ -3
\end{array} \right]$
under the transformation $T$.
\end{parts}
\vspace{\stretch{4}}


\clearpage
\question[12]
\begin{parts}
\part
Let $A$ be a square matrix of size $n \times n$.  Give the definition of the inverse matrix of $A$.  
\vspace{\stretch{1}}

\part
Find the inverse of the matrix $A = 
\left[
\begin{array}{cc}
	1 & 3 \\ 
    2 & 5
\end{array}
\right]$, or show that the inverse does not exist.
\end{parts}
\vspace{\stretch{3}}

\end{questions}

\end{document}