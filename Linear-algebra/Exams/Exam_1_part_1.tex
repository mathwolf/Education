\documentclass[addpoints, 12pt]{exam}

\setlength{\topmargin}{-.75in} \addtolength{\textheight}{2.00in}
\setlength{\oddsidemargin}{.00in} \addtolength{\textwidth}{.25in}

\usepackage[inline]{asymptote}
\usepackage{amsmath, gensymb, nicefrac, verbatim, graphicx, multicol}
\usepackage[export]{adjustbox}

\nofiles

\pagestyle{empty}

\setlength{\parindent}{0in}

\title{Linear Exam 1 part 1 - Fall 2016}
\author{Wray}
\begin{document}

\begin{asydef}
//
// Global Asymptote definitions can be put here.
//
usepackage("bm");
texpreamble("\def\V#1{\bm{#1}}");
\end{asydef}

\noindent {\sc {\bf {\Large Exam 1, part 1}}
            \hfill MAT 255, Fall 2016}
\bigskip

\noindent {\sc  {\large Name:}
             \hfill}
             
\bigskip
Solve the following problems without using a calculator.

\begin{questions}

\question[15]
Solve the system using Gaussian elimination.  Describe the row operation you are using at each step.  (Other methods of solving this problem will not be given credit.)
\begin{gather*}
	2x + y + z = 22 \\
    x - y = 11 \\
    y - z = -4
\end{gather*}

\clearpage
\question[10]
Define the following matrices.
\begin{align*}
A & = \left[ \begin{array}{c}
	4 \\ 2 \\ 6
\end{array}
\right]
& B & = \left[ \begin{array}{cc}
	-1 & 1 \\ 
    2 & 0
\end{array}
\right] 
& C & = \left[ \begin{array}{ccc}
	-6 & 3 & -3
\end{array}
\right] 
& D & = \left[ \begin{array}{ccc}
	-5 & 4 & 7 \\ 
    -1 & 3 & 6
\end{array}
\right]
\end{align*}
\bigskip
\begin{parts}
\part
Based on these definitions, describe each of the following products as defined or undefined.
\renewcommand{\labelenumi}{\roman{enumi}.}
\begin{multicols}{2}
\begin{enumerate}
\item $DA$
\bigskip
\bigskip
\item $DC$
\bigskip
\bigskip
\item $BD$
\bigskip
\bigskip
\columnbreak
\item $DB$
\bigskip
\bigskip
\item $AC$
\bigskip
\bigskip
\item $CA$
\bigskip
\bigskip
\end{enumerate}
\end{multicols}
\bigskip
\bigskip

\part
Choose ONE of the answers from part (a) that you said was defined.  Calculate the matrix product.
\vspace{\stretch{2}}
\end{parts}

\end{questions}

\end{document}