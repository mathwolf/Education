\documentclass[addpoints, 12pt]{exam}

\setlength{\topmargin}{-.75in} \addtolength{\textheight}{2.00in}
\setlength{\oddsidemargin}{.00in} \addtolength{\textwidth}{.25in}

\usepackage{amsmath,color,graphicx, multicol, nicefrac}
\usepackage[inline]{asymptote}

\nofiles

\setlength{\parindent}{0in}

\pagestyle{plain}

\title{Quiz template}
\author{Wray}
\begin{document}

\begin{asydef}
//
// Global Asymptote definitions can be put here.
//
usepackage("bm");
texpreamble("\def\V#1{\bm{#1}}");
\end{asydef}



\noindent {\sc {\bf {\Large Quiz 5B: Basic graphs}}
            \hfill Math 1110, Fall 2017}
\bigskip

\noindent {\sc  {\large Name:}
             \hfill}
             
\bigskip
\bigskip

\smallskip

Graph each of the functions.  Your graph should show the correct shape and the coordinates of two or three important points.

\begin{questions}

\begin{multicols}{2}
\question
$f(x) = x^{\nicefrac{1}{3}}$

\vspace{\stretch{1}}

\begin{asy}
size(7cm);
import graph;
for (int i = -5; i <= 5; ++i)
	{
    draw((-5,i)--(5,i), gray);
    draw((i,-5)--(i,5), gray);
    }
draw(Label("$x$", EndPoint, E), (-6,0)--(6,0), Arrows);
draw(Label("$y$", EndPoint, N), (0,-6)--(0,6), Arrows);
\end{asy}

\vspace{\stretch{1}}

\question[5]
$f(x) = x^{5}$

\vspace{\stretch{1}}

\begin{asy}
size(7cm);
import graph;
for (int i = -5; i <= 5; ++i)
	{
    draw((-5,i)--(5,i), gray);
    draw((i,-5)--(i,5), gray);
    }
draw(Label("$x$", EndPoint, E), (-6,0)--(6,0), Arrows);
draw(Label("$y$", EndPoint, N), (0,-6)--(0,6), Arrows);
\end{asy}

\vspace{\stretch{1}}

\end{multicols} 

\vspace{\stretch{2}}

\begin{multicols}{2}
\question
$f(x) = x^{-4}$

\vspace{\stretch{1}}

\begin{asy}
size(7cm);
import graph;
for (int i = -5; i <= 5; ++i)
	{
    draw((-5,i)--(5,i), gray);
    draw((i,-5)--(i,5), gray);
    }
draw(Label("$x$", EndPoint, E), (-6,0)--(6,0), Arrows);
draw(Label("$y$", EndPoint, N), (0,-6)--(0,6), Arrows);
\end{asy}

\vspace{\stretch{1}}

\question[5]
$f(x) = x^{6}$

\vspace{\stretch{1}}

\begin{asy}
size(7cm);
import graph;
for (int i = -5; i <= 5; ++i)
	{
    draw((-5,i)--(5,i), gray);
    draw((i,-5)--(i,5), gray);
    }
draw(Label("$x$", EndPoint, E), (-6,0)--(6,0), Arrows);
draw(Label("$y$", EndPoint, N), (0,-6)--(0,6), Arrows);
\end{asy}

\vspace{\stretch{1}}

\end{multicols} 

\vspace{\stretch{3}}

\clearpage

\begin{multicols}{2}
\question
$f(x) = x^{\nicefrac{1}{4}}$

\vspace{\stretch{1}}

\begin{asy}
size(7cm);
import graph;
for (int i = -5; i <= 5; ++i)
	{
    draw((-5,i)--(5,i), gray);
    draw((i,-5)--(i,5), gray);
    }
draw(Label("$x$", EndPoint, E), (-6,0)--(6,0), Arrows);
draw(Label("$y$", EndPoint, N), (0,-6)--(0,6), Arrows);
\end{asy}

\vspace{\stretch{1}}

\question[5]
$f(x) = \left( x + 2 \right)^{\nicefrac{1}{4}}$

\vspace{\stretch{1}}

\begin{asy}
size(7cm);
import graph;
for (int i = -5; i <= 5; ++i)
	{
    draw((-5,i)--(5,i), gray);
    draw((i,-5)--(i,5), gray);
    }
draw(Label("$x$", EndPoint, E), (-6,0)--(6,0), Arrows);
draw(Label("$y$", EndPoint, N), (0,-6)--(0,6), Arrows);
\end{asy}

\vspace{\stretch{1}}

\end{multicols} 

\vspace{\stretch{2}}

\begin{multicols}{2}
\question
$f(x) = x^{-3}$

\vspace{\stretch{1}}

\begin{asy}
size(7cm);
import graph;
for (int i = -5; i <= 5; ++i)
	{
    draw((-5,i)--(5,i), gray);
    draw((i,-5)--(i,5), gray);
    }
draw(Label("$x$", EndPoint, E), (-6,0)--(6,0), Arrows);
draw(Label("$y$", EndPoint, N), (0,-6)--(0,6), Arrows);
\end{asy}

\vspace{\stretch{1}}

\question[5]
$f(x) = x^{-3} + 2$

\vspace{\stretch{1}}

\begin{asy}
size(7cm);
import graph;
for (int i = -5; i <= 5; ++i)
	{
    draw((-5,i)--(5,i), gray);
    draw((i,-5)--(i,5), gray);
    }
draw(Label("$x$", EndPoint, E), (-6,0)--(6,0), Arrows);
draw(Label("$y$", EndPoint, N), (0,-6)--(0,6), Arrows);
\end{asy}

\vspace{\stretch{1}}

\end{multicols} 

\vspace{\stretch{2}}

\end{questions}

\end{document}