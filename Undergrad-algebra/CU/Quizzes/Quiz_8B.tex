\documentclass[addpoints, 12pt]{exam}

\setlength{\topmargin}{-.75in} \addtolength{\textheight}{2.00in}
\setlength{\oddsidemargin}{.00in} \addtolength{\textwidth}{.25in}

\usepackage{amsmath,color,graphicx, multicol, nicefrac}
\usepackage[inline]{asymptote}

\nofiles

\setlength{\parindent}{0in}

\pagestyle{plain}

\title{Quiz template}
\author{Wray}
\begin{document}

\begin{asydef}
//
// Global Asymptote definitions can be put here.
//
usepackage("bm");
texpreamble("\def\V#1{\bm{#1}}");
\end{asydef}



\noindent {\sc {\bf {\Large Quiz 8B: Inverse functions, exponents}}
            \hfill Fall 2017}
\bigskip

\noindent {\sc  {\large Name:}
             \hfill}
             
\bigskip
\bigskip

\begin{questions}

\question[10]
Find the inverse of the function.
\begin{align*}
f(x) = \dfrac{4x - 3}{2 - x}
\end{align*}
You may assume that this function is one-to-one without testing it.

\newpage 

\question[10]
Sketch a picture of the function 
\begin{align*}
y = 3 \cdot 2^{x + 1}
\end{align*}
by using transformations.  Show the coordinates of two or three important points on your graph.

\begin{center}
\begin{asy}
size(12cm);
for (int i = -6; i <= 6; ++i)
	{
    draw((-6,2 * i)--(6,2 * i), gray);
    draw((-6,2 * i + 1)--(6,2 * i + 1), gray);
    draw((i,-12)--(i,13), gray);
    }
\end{asy}
\end{center}

\end{questions}

\end{document}