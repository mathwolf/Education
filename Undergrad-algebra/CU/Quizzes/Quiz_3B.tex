\documentclass[addpoints, 12pt]{exam}

\setlength{\topmargin}{-.75in} \addtolength{\textheight}{2.00in}
\setlength{\oddsidemargin}{.00in} \addtolength{\textwidth}{.25in}

\usepackage{amsmath,color,graphicx, multicol}
\usepackage[inline]{asymptote}

\nofiles

\setlength{\parindent}{0in}

\pagestyle{plain}

\title{Quiz template}
\author{Wray}
\begin{document}

\begin{asydef}
//
// Global Asymptote definitions can be put here.
//
usepackage("bm");
texpreamble("\def\V#1{\bm{#1}}");
\end{asydef}



\noindent {\sc {\bf {\Large Quiz 3B: Radicals, lines}}
            \hfill Math 1110, Fall 2017}
\bigskip

\noindent {\sc  {\large Name:}
             \hfill}
             
\bigskip
\bigskip

\smallskip

\begin{questions}

\question[3]
Solve $\sqrt{x+3} = 4$.

\vspace{\stretch{2}}

\question[5]
Solve $x = \sqrt{3x + 10}$.

\vspace{\stretch{2}}

\question[2]
Graph the line $5x - 3y = 9$.

\bigskip

\begin{asy}
size(6cm);
for (int i = -6; i <= 6; ++i)
	{
    draw((-6,i)--(6,i), gray);
    draw((i,-6)--(i,6), gray);
    }
draw(Label("$x$", EndPoint, E), (-7,0)--(7,0), Arrows);
draw(Label("$y$", EndPoint, N), (0,-7)--(0,7), Arrows);
\end{asy}

\vspace{\stretch{1}}

\clearpage 

\question[10]
Consider the points $(2,-2)$ and $(-2,1)$.

\begin{parts}
\part
Find the distance between these two points.

\vspace{\stretch{2}}

\part 
Find the slope of the line containing these points.

\vspace{\stretch{2}}

\part
Find the equation of the line containing these points.

\vspace{\stretch{2}}

\part
Graph the equation that you found in part (c)

\bigskip

\begin{asy}
size(6cm);
for (int i = -6; i <= 6; ++i)
	{
    draw((-6,i)--(6,i), gray);
    draw((i,-6)--(i,6), gray);
    }
draw(Label("$x$", EndPoint, E), (-7,0)--(7,0), Arrows);
draw(Label("$y$", EndPoint, N), (0,-7)--(0,7), Arrows);
\end{asy}

\vspace{\stretch{1}}

\end{parts}

\end{questions}

\end{document}