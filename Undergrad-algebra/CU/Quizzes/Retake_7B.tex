\documentclass[addpoints, 12pt]{exam}

\setlength{\topmargin}{-.75in} \addtolength{\textheight}{2.00in}
\setlength{\oddsidemargin}{.00in} \addtolength{\textwidth}{.25in}

\usepackage{amsmath,color,graphicx, multicol, nicefrac}
\usepackage[inline]{asymptote}

\nofiles

\setlength{\parindent}{0in}

\pagestyle{plain}

\title{Quiz template}
\author{Wray}
\begin{document}

\begin{asydef}
//
// Global Asymptote definitions can be put here.
//
usepackage("bm");
texpreamble("\def\V#1{\bm{#1}}");
\end{asydef}



\noindent {\sc {\bf {\Large Retake 7B: Complex zeros, rational functions}}
            \hfill Fall 2017}
\bigskip

\noindent {\sc  {\large Name:}
             \hfill}
             
\bigskip
\bigskip

\begin{questions}

\question[10]
Define $f(x) = 3x^4 + 5x^3 + 25x^2 +45x - 18$.  One zero of this function is $x = -3i$.  Find all remaining zeros, both real and complex.

\newpage 

\question
Consider the rational function.
\begin{align*}
R(x) = \dfrac{2x^2 + 4x + 2}{x^2 + x - 6}
\end{align*}

\begin{parts}
\part 
Analyze the ratio of leading terms to find the end behavior.  If the function has a horizontal asymptote or a slant asymptote, give its equation.

\vspace{\stretch{3}}

\part 
Find the $x$-intercepts of the graph, if any.

\vspace{\stretch{2}}

\part 
Find the equations for the vertical asymptotes, if any.

\vspace{\stretch{2}}

\part 
Give the $x$-coordinates for any holes that you identify.

\vspace{\stretch{2}}

\part
Based on your work on the previous parts, give the $x-$ and $y$-coordinates for an appropriate selection of test points.

\vspace{\stretch{3}}

\part
Graph the function

\vspace{\stretch{1}}
\begin{center}
\begin{asy}
size(6cm);
for (int i = -8; i <= 8; ++i)
	{
    draw((-8,i)--(8,i), gray);
    draw((i,-8)--(i,8), gray);
    }
\end{asy}
\end{center}
\vspace{\stretch{1}}

\end{parts}

\end{questions}

\end{document}