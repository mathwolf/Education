\documentclass[addpoints, 12pt]{exam}

\setlength{\topmargin}{-.75in} \addtolength{\textheight}{2.00in}
\setlength{\oddsidemargin}{.00in} \addtolength{\textwidth}{.25in}

\usepackage[inline]{asymptote}
\usepackage{amsmath, gensymb, nicefrac, verbatim, graphicx}
\usepackage[export]{adjustbox}

\nofiles

\pagestyle{empty}

\setlength{\parindent}{0in}

\title{Algebra Exam 1 review guide - CU Fall 2017}
\author{Wray}
\begin{document}

\begin{asydef}
//
// Global Asymptote definitions can be put here.
//
usepackage("bm");
texpreamble("\def\V#1{\bm{#1}}");
\end{asydef}

\noindent {\sc {\bf {\Large Exam 1 review guide}}
            \hfill MAT 1110, Fall 2017}
\bigskip             
\bigskip

Our first exam is Wednesday, September 20.  It covers the material from the first four homework assignments.

\subsection*{Fundamentals}

These topics are basic for doing algebra at the level of our class.  There is not enough time to include everything on the test, but I expect anyone who gets a passing grade of C or higher to be ready for all the problems in this section.
\bigskip

Solve a simple equation (including the possibility of ``no solutions'' and ``all numbers'')

Factor to solve a quadratic equation

Complete the square to solve a quadratic equation

Equations with square roots

Graphs and equations of lines

Graphs and equations of circles

Determine if a graph shows a function by using the vertical line test

Graph a function using a table

For a graph, describe the following properties: domain, range, and intercepts

For a function given as an equation, find the domain

Describe the intervals where the graph of a function is increasing, decreasing, and constant

Find maximum and minimum values of a function on a graph

Determine if a graph has symmetry with respect to the $x$-axis, the $y$-axis, or the origin
\bigskip

In particular, make sure you look at the definition of parallel/perpendicular lines and horizontal/vertical lines.  I posted practice problems for these topics under ``Files'' on our Canvas page.  Some of these are the same problems we used in class.
\bigskip

We will look at some additional review materials for graphs and symmetry in class on Monday.

\subsection*{In-depth topics}

To get an A, you should be able to solve the following additional problems.
\bigskip

Analyze an equation for symmetry

Simplify a difference quotient

Find the average rate of change for a function

Problems that combine two or more basic problems on the list above
\bigskip

One example of a combination problem is a quadratic equation that has ``no solutions'' or ``all numbers'' as the solution.  The same idea applies to square root equations.




\end{document}