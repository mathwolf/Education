\documentclass[addpoints, 12pt]{exam}

\setlength{\topmargin}{-.75in} \addtolength{\textheight}{2.00in}
\setlength{\oddsidemargin}{.00in} \addtolength{\textwidth}{.25in}

\usepackage[inline]{asymptote}
\usepackage{amsmath, gensymb, nicefrac, verbatim, graphicx}
\usepackage[export]{adjustbox}

\nofiles

\pagestyle{empty}

\setlength{\parindent}{0in}

\title{Algebra Exam 2 review guide - CU Fall 2017}
\author{Wray}
\begin{document}

\begin{asydef}
//
// Global Asymptote definitions can be put here.
//
usepackage("bm");
texpreamble("\def\V#1{\bm{#1}}");
\end{asydef}

\noindent {\sc {\bf {\Large Exam 2 review guide}}
            \hfill MAT 1110, Fall 2017}
\bigskip             
\bigskip

The second exam is Wednesday, October 18.  It covers the material on homework assignments 5 through 8.

\subsection*{Fundamentals}

Sketch the basic graph for a power of $x$ without a calculator, including fractions and negative powers \smallskip

Combine these basic graphs with the rules for transformations: horizontal and vertical shifts, reflection, stretching and shrinking \smallskip

Evaluate or graph a function defined by a piecewise formula

Write a piecewise formula for a graph made of two lines

Solve problems involving quadratic functions, including a problem with no solutions

Graph a parabola, and identify the vertex, intercepts, and axis of symmetry

Linear and quadratic regression using the calculator

Describe the degree and leading term of a polynomial

Multiply and divide polynomials

Factor a polynomial of degree 3 or greater, and then give the zeros and their multiplicities

Analyze a graph of a polynomial function

\subsection*{Advanced topics}

Solve an application based on a quadratic function and identify any minimum or maximum values \smallskip 

Graph a polynomial of degree 3 or greater

\end{document}