\documentclass[addpoints, 12pt]{exam}

\setlength{\topmargin}{-.75in} \addtolength{\textheight}{2.00in}
\setlength{\oddsidemargin}{.00in} \addtolength{\textwidth}{.25in}

\usepackage[inline]{asymptote}
\usepackage{amsmath, gensymb, nicefrac, verbatim, graphicx}
\usepackage[export]{adjustbox}

\nofiles

\pagestyle{empty}

\setlength{\parindent}{0in}

\title{Algebra Exam 1 Part 2 - CU Fall 2017}
\author{Wray}
\begin{document}

\begin{asydef}
//
// Global Asymptote definitions can be put here.
//
usepackage("bm");
texpreamble("\def\V#1{\bm{#1}}");
\end{asydef}

\noindent {\sc {\bf {\Large Exam 1 review question}}
            \hfill Math 1110, Fall 2017}
\bigskip

\noindent {\sc  {\large Name:}
             \hfill}
             
\bigskip
Use a calculator to answer the following question.

\begin{questions}

\question[10]
\begin{parts}

\part 
Use a table to graph the following function.
\begin{align*}
g(x) = \sqrt{x^2 - 6x + 11}
\end{align*}
Your table should give $(x,y)$ pairs for at least six points.  Show both positive and negative values of $x$.

\bigskip

\begin{asy}
size(8cm);
for (int i = -6; i <= 6; ++i)
	{
    draw((-6,i)--(6,i), gray);
    draw((i,-6)--(i,6), gray);
    }
draw(Label("$x$", EndPoint, E), (-7,0)--(7,0), Arrows);
draw(Label("$y$", EndPoint, N), (0,-7)--(0,7), Arrows);
\end{asy}

\vspace{\stretch{1}}

\part
Find the average rate of change for $g(x)$ on the interval between $x = -2$ and $x = 2$.

\vspace{\stretch{2}}

\end{parts}

\end{questions}

\end{document}