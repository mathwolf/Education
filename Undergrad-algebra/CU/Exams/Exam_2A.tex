\documentclass[addpoints, 12pt]{exam}

\setlength{\topmargin}{-.75in} \addtolength{\textheight}{2.00in}
\setlength{\oddsidemargin}{.00in} \addtolength{\textwidth}{.25in}

\usepackage[inline]{asymptote}
\usepackage{amsmath, gensymb, nicefrac, verbatim, graphicx, multicol}
\usepackage[export]{adjustbox}

\nofiles

\pagestyle{empty}

\setlength{\parindent}{0in}

\title{Algebra Exam 2A Part 1 - CU Fall 2017}
\author{Wray}
\begin{document}

\begin{asydef}
//
// Global Asymptote definitions can be put here.
//
usepackage("bm");
texpreamble("\def\V#1{\bm{#1}}");
\end{asydef}

\noindent {\sc {\bf {\Large Exam 2A: Part 1}}
            \hfill Math 1110, Fall 2017}
\bigskip

\noindent {\sc  {\large Name:}
             \hfill}
             
\bigskip
You can use a calculator on this part of the exam.

\begin{questions}

\question[16]
Define the function $f(x) = -5x^2 + 3x + 7$.
\begin{parts}
\part
Evaluate $f(3.5)$.

\vspace{\stretch{2}}

\part 
Solve $f(x) = 0$.  Show the formula that you use to get your answer.

\vspace{\stretch{3}}

\part 
Solve $f(x) = 10$

\vspace{\stretch{4}}

\end{parts}

\clearpage 

\question[24]
A rocket is launched at time $t=0$.  Its height above the ground is given by the equation \begin{equation*}
	y(t) = -6t^2 + 200t + 1200,
\end{equation*}
where $y$ is height (in meters) and $t$ is time (in seconds).
\begin{parts}
\part
How high is the rocket at the time it is launched?

\vspace{\stretch{2}}

\part
How high is the rocket after 15 seconds?

\vspace{\stretch{2}}

\part
What is the maximum height of the rocket?  Show the formula that you use to find your answer.

\vspace{\stretch{3}}

\part
At what time does the rocket hit the ground?

\vspace{\stretch{4}}

\end{parts}

\end{questions}

\clearpage 

\noindent {\sc {\bf {\Large Exam 2A: Part 2}}
            \hfill Math 1110, Fall 2017}
\bigskip

\noindent {\sc  {\large Name:}
             \hfill}
             
\bigskip
No calculators on this part of the exam.  
\bigskip

\begin{questions}
\setcounter{question}{3}

\question[12]
Graph each of the functions.  Your graph should show the correct shape and the coordinates of two or three important points.

\begin{parts}
\begin{multicols}{2}
\part
$f(x) = x^{4}$

\vspace{\stretch{1}}

\begin{asy}
size(7cm);
import graph;
for (int i = -5; i <= 5; ++i)
	{
    draw((-5,i)--(5,i), gray);
    draw((i,-5)--(i,5), gray);
    }
draw(Label("$x$", EndPoint, E), (-6,0)--(6,0), Arrows);
draw(Label("$y$", EndPoint, N), (0,-6)--(0,6), Arrows);
\end{asy}

\vspace{\stretch{1}}

\part
$f(x) = x^3 - \dfrac{3}{2}$

\vspace{\stretch{1}}

\begin{asy}
size(7cm);
import graph;
for (int i = -5; i <= 5; ++i)
	{
    draw((-5,i)--(5,i), gray);
    draw((i,-5)--(i,5), gray);
    }
draw(Label("$x$", EndPoint, E), (-6,0)--(6,0), Arrows);
draw(Label("$y$", EndPoint, N), (0,-6)--(0,6), Arrows);
\end{asy}

\vspace{\stretch{1}}

\end{multicols} 

\vspace{\stretch{2}}

\begin{multicols}{2}
\part
$f(x) = x^{\nicefrac{1}{2}}$

\vspace{\stretch{1}}

\begin{asy}
size(7cm);
import graph;
for (int i = -5; i <= 5; ++i)
	{
    draw((-5,i)--(5,i), gray);
    draw((i,-5)--(i,5), gray);
    }
draw(Label("$x$", EndPoint, E), (-6,0)--(6,0), Arrows);
draw(Label("$y$", EndPoint, N), (0,-6)--(0,6), Arrows);
\end{asy}

\vspace{\stretch{1}}

\part
$f(x) = x^{\nicefrac{1}{3}}$

\vspace{\stretch{1}}

\begin{asy}
size(7cm);
import graph;
for (int i = -5; i <= 5; ++i)
	{
    draw((-5,i)--(5,i), gray);
    draw((i,-5)--(i,5), gray);
    }
draw(Label("$x$", EndPoint, E), (-6,0)--(6,0), Arrows);
draw(Label("$y$", EndPoint, N), (0,-6)--(0,6), Arrows);
\end{asy}

\vspace{\stretch{1}}

\end{multicols} 

\vspace{\stretch{3}}

\clearpage

\begin{multicols}{2}
\part
$f(x) = x^{-3}$

\vspace{\stretch{1}}

\begin{asy}
size(7cm);
import graph;
for (int i = -5; i <= 5; ++i)
	{
    draw((-5,i)--(5,i), gray);
    draw((i,-5)--(i,5), gray);
    }
draw(Label("$x$", EndPoint, E), (-6,0)--(6,0), Arrows);
draw(Label("$y$", EndPoint, N), (0,-6)--(0,6), Arrows);
\end{asy}

\vspace{\stretch{1}}

\part
$f(x) = \left( 2 - x \right)^{-3}$

\vspace{\stretch{1}}

\begin{asy}
size(7cm);
import graph;
for (int i = -5; i <= 5; ++i)
	{
    draw((-5,i)--(5,i), gray);
    draw((i,-5)--(i,5), gray);
    }
draw(Label("$x$", EndPoint, E), (-6,0)--(6,0), Arrows);
draw(Label("$y$", EndPoint, N), (0,-6)--(0,6), Arrows);
\end{asy}

\vspace{\stretch{1}}

\end{multicols} 
\end{parts}

\vspace{\stretch{2}}

\question[3]
Define the piecewise function 
\begin{equation*}
g(x) = 
\begin{cases}
3x^2 - 1 &\mbox{if } x < 2 \\
5 - x &\mbox{if } x \ge 2
\end{cases}
\end{equation*}

\begin{parts}
\part
Evaluate $g(0)$.

\vspace{\stretch{2}}

\part
Evaluate $g(5)$.
\end{parts}

\vspace{\stretch{3}}

\clearpage 

\question[10]
Let $y = -2x^2 - 4x + 1$. \smallskip
\begin{parts}
\part 
Graph the parabola. \newline
  \begin{asy}
  size(7cm);
  import graph;
  for (int i = -5; i <= 5; ++i)
      {
      draw((-5,i)--(5,i), gray);
      draw((i,-5)--(i,5), gray);
      }
  draw(Label("$x$", EndPoint, E), (-6,0)--(6,0), Arrows);
  draw(Label("$y$", EndPoint, N), (0,-6)--(0,6), Arrows);
  \end{asy}

\part 
Give the coordinates of the vertex.

\vspace{\stretch{1}}

\end{parts}

\question[10]
Find $(4x^3 - 3x^2 -8x + 4) \div (x-2)$.

\vspace{\stretch{4}}

\question[3]
Give a polynomial function that has zeros -9, 0, 1 and 14.

\vspace{\stretch{2}}

\clearpage 

\question[22]
Consider the polynomial function $f(x) = - x^3 + 4 x^2 + 4x - 16$.
\begin{parts}
\part
Give the degree and leading term.

\vspace{\stretch{1}}

\part
Factor completely and give all of the zeros.
\vspace{\stretch{5}}

\part
Sketch a graph of the function.

\vspace{\stretch{1}}

\begin{asy}
size(8cm);
for (int i = -10; i <= 10; ++i)
	{
    draw((-10,i)--(10,i), mediumgray);
    draw((i,-10)--(i,10), mediumgray);
    }
\end{asy}

\vspace{\stretch{1}}
\end{parts}

\end{questions}

\end{document}