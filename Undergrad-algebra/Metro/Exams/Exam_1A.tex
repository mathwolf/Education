\documentclass[addpoints, 12pt]{exam}

\setlength{\topmargin}{-.75in} \addtolength{\textheight}{2.00in}
\setlength{\oddsidemargin}{.00in} \addtolength{\textwidth}{.25in}

\usepackage[inline]{asymptote}
\usepackage{amsmath, gensymb, nicefrac, verbatim, graphicx}
\usepackage[export]{adjustbox}

\nofiles

\pagestyle{empty}

\setlength{\parindent}{0in}

\title{Algebra Exam 1 - Metro Fall 2017}
\author{Wray}
\begin{document}

\begin{asydef}
//
// Global Asymptote definitions can be put here.
//
usepackage("bm");
texpreamble("\def\V#1{\bm{#1}}");
\end{asydef}

\noindent {\sc {\bf {\Large Exam 1A}}
            \hfill MAT 1110, Fall 2017}
\bigskip

\noindent {\sc  {\large Name:}
             \hfill}
             
\bigskip

\begin{questions}

\question[2]
Consider the following equation.
\begin{align*}
2 - x^2 = x^3  
\end{align*}
Is the number $x = -1$ a solution to this equation?  Show the steps that support your answer.

\vspace{\stretch{2}}

\question[8]
Solve each equation.

\begin{parts}
\part
$5x - 2 = 5$

\vspace{\stretch{2}}

\part
$3(x-1) = 5 - 2(3x+4)$

\vspace{\stretch{2}}

\part
$-4(x - 2) = 2 + 2(3 - 2x)$

\vspace{\stretch{3}}

\end{parts}

\clearpage

\question[11]
Use factoring to solve each of the following quadratic equations.

\begin{parts}
\part
$x^2 - 25 = 0$

\vspace{\stretch{2}}

\part
$x^2 + 15 = 8x$

\vspace{\stretch{2}}

\part
$9x^2 - 4x = 0$

\vspace{\stretch{2}}

\part
$12x^2 - 5x - 3 = 0$

\vspace{\stretch{3}}

\end{parts}

\newpage

\question[9]
Use completing the square to solve $x^2 + 28 = - 10x$.

\newpage

\question[10]
Solve each equation.  Remember to take steps to address the possibility of false solutions.

\begin{parts}
\part 
$5 = \sqrt{3 - n}$

\vspace{\stretch{2}}

\part  
$\sqrt{3x + 10} = -16$

\vspace{\stretch{2}}

\part 
$x = \sqrt{30 - x}$

\vspace{\stretch{3}}

\end{parts}

\clearpage

\question[12]
Solve each inequality.  Write your answer in interval notation, and make a number line that shows a picture of your solutions.

\begin{parts}
\part
$3 \left( y + 2 \right) - 5y \le -4$

\vspace{\stretch{2}}

\part
$x + 1 > x + 2$

\vspace{\stretch{2}}

\part
$\big\lvert 4x - 7 \big\rvert - 10 \le -5$

\vspace{\stretch{3}}

\end{parts}

\clearpage

\question[10]
Suppose that copper costs $\$7.50$ per kilogram, and tin costs $\$10.00$ per kilogram.  We want to make
$16.5$ kilograms of bronze by combining the two metals.  The cost of the mixture should be $\$8.50$ per kilogram.  How much of the two metals should we use?

\clearpage

\question[2]
Find the equation of the line with slope $- \nicefrac{7}{4}$ and $y$-intercept $12$.

\vspace{\stretch{2}}

\question[13]
Consider the points $(2,-4)$ and $(-4,4)$.

\begin{parts}
\part 
Find the distance between the points.

\vspace{\stretch{2}}

\part
Find the slope of the line that contains these points.

\vspace{\stretch{2}}

\part
Find the equation of the line that contains these points.

\vspace{\stretch{2}}

\part
Graph the line that you found in part (c).

\smallskip

\begin{asy}
size(8cm);
for (int i = -6; i <= 6; ++i)
	{
    draw((-6,i)--(6,i), gray);
    draw((i,-6)--(i,6), gray);
    }
draw(Label("$x$", EndPoint, E), (-7,0)--(7,0), Arrows);
draw(Label("$y$", EndPoint, N), (0,-7)--(0,7), Arrows);
\end{asy}

\vspace{\stretch{1}}

\end{parts}

\newpage

\question[7]

\begin{parts}

\part
Graph the line $3x - 5y = -10$.

\smallskip

\begin{asy}
size(8cm);
for (int i = -6; i <= 6; ++i)
	{
    draw((-6,i)--(6,i), gray);
    draw((i,-6)--(i,6), gray);
    }
draw(Label("$x$", EndPoint, E), (-7,0)--(7,0), Arrows);
draw(Label("$y$", EndPoint, N), (0,-7)--(0,7), Arrows);
\end{asy}

\vspace{\stretch{1}}

\part
On the same set of axes, sketch a graph of the line that is parallel to your line from part (a) and that passes through the point $(-1,-4)$.

\vspace{\stretch{2}}

\part
Write the equation of the parallel line that you graphed in part (b).

\vspace{\stretch{3}}

\end{parts}

\newpage

\question[7]

\begin{parts}

\part
Graph the line $y = -3$.

\smallskip

\begin{asy}
size(8cm);
for (int i = -6; i <= 6; ++i)
	{
    draw((-6,i)--(6,i), gray);
    draw((i,-6)--(i,6), gray);
    }
draw(Label("$x$", EndPoint, E), (-7,0)--(7,0), Arrows);
draw(Label("$y$", EndPoint, N), (0,-7)--(0,7), Arrows);
\end{asy}

\vspace{\stretch{1}}

\part
On the same set of axes, sketch a graph of the line that is perpendicular to your line from part (a) and that passes through the point $(-1,3)$.

\vspace{\stretch{2}}

\part
Write the equation of the perpendicular line that you graphed in part (b).

\vspace{\stretch{3}}

\end{parts}

\newpage

\question[4]
Find the equation of the circle that has center $(-200,450)$ and radius $100$.

\vspace{\stretch{3}}

\question[5]
Graph the equation $(x - 3)^2 + (y + 1)^2 = \dfrac{25}{4}$.


\bigskip
\bigskip

\begin{asy}
size(8cm);
for (int i = -6; i <= 6; ++i)
	{
    draw((-6,i)--(6,i), gray);
    draw((i,-6)--(i,6), gray);
    }
draw(Label("$x$", EndPoint, E), (-7,0)--(7,0), Arrows);
draw(Label("$y$", EndPoint, N), (0,-7)--(0,7), Arrows);
\end{asy}

\vspace{\stretch{2}}

\end{questions}

\end{document}