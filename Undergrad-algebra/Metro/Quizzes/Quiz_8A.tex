\documentclass[addpoints, 12pt]{exam}

\setlength{\topmargin}{-.75in} \addtolength{\textheight}{2.00in}
\setlength{\oddsidemargin}{.00in} \addtolength{\textwidth}{.25in}

\usepackage{amsmath,color,graphicx, multicol, nicefrac}
\usepackage[inline]{asymptote}

\nofiles

\setlength{\parindent}{0in}

\pagestyle{plain}

\title{Quiz template}
\author{Wray}
\begin{document}

\begin{asydef}
//
// Global Asymptote definitions can be put here.
//
usepackage("bm");
texpreamble("\def\V#1{\bm{#1}}");
\end{asydef}



\noindent {\sc {\bf {\Large Quiz 8A: Rational functions}}
            \hfill Math 1110, Fall 2017}
\bigskip

\noindent {\sc  {\large Name:}
             \hfill}
             
\bigskip
\bigskip

\begin{questions}

\question[10]
Consider the rational function.
\begin{align*}
R(x) = \dfrac{2x^2 + 8x}{x^2 + 2x - 3}
\end{align*}

\begin{parts}
\part 
Analyze the ratio of leading terms to find the end behavior.  If the function has a horizontal asymptote or a slant asymptote, give its equation.

\vspace{\stretch{3}}

\part 
Find the $x$-intercepts of the graph, if any.

\vspace{\stretch{2}}

\part 
Find the equations for the vertical asymptotes, if any.

\vspace{\stretch{2}}

\part 
Give the $x$-coordinates for any holes that you identify.

\vspace{\stretch{2}}

\part
Based on your work on the previous parts, give the $x-$ and $y$-coordinates for an appropriate selection of test points.

\vspace{\stretch{3}}

\part
Graph the function

\vspace{\stretch{1}}
\begin{center}
\begin{asy}
size(6cm);
for (int i = -8; i <= 8; ++i)
	{
    draw((-8,i)--(8,i), gray);
    draw((i,-8)--(i,8), gray);
    }
\end{asy}
\end{center}
\vspace{\stretch{1}}

\end{parts}

\clearpage 

\question[10]
Write an equation for the rational function shown in the graph.  The asymptotes are shown as dashed lines.
\vspace{\stretch{1}}

\begin{center}
\begin{asy}
size(10cm);
import graph;
for (int i = -6; i <= -3; ++i)
	{
    draw((i,-6)--(i,6), gray);
    }
for (int i = -1; i <= 0; ++i)
	{
    draw((i,-6)--(i,6), gray);
    }
for (int i = 2; i <= 6; ++i)
	{
    draw((i,-6)--(i,6), gray);
    }
    
for (int i = -6; i <= -2; ++i)
	{
    draw((-6,i)--(6,i), gray);
    }
for (int i = 0; i <= 6; ++i)
	{
    draw((-6,i)--(6,i), gray);
    }
draw(Label("$x$", EndPoint, E), (-7,0)--(7,0), Arrows);
draw(Label("$y$", EndPoint, N), (0,-7)--(0,7), Arrows);

draw((-2,7)--(-2,-7), dashed, Arrows);
draw((1,7)--(1,-7), dashed, Arrows);
draw((-7,-1)--(7,-1), dashed, Arrows);

pair F(real t) { 
	return (t, -((t + 1)^2 * (t + 4)) / ((t + 2)^2 * (t - 1)) );
}

path g = graph(F, -5.9, -2.4, operator ..);
draw(g, linewidth(1));
path g = graph(F, -1.725, 0.68, operator ..);
draw(g, linewidth(1));
path g = graph(F, 1.47, 5.9, operator ..);
draw(g, linewidth(1));

dotfactor *= 2;
dot((-1,0)); 
dot((-4,0));

\end{asy}
\end{center}
\end{questions}
\vspace{\stretch{16}}

\end{document}