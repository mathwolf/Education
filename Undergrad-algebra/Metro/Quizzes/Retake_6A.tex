\documentclass[addpoints, 12pt]{exam}

\setlength{\topmargin}{-.75in} \addtolength{\textheight}{2.00in}
\setlength{\oddsidemargin}{.00in} \addtolength{\textwidth}{.25in}

\usepackage{amsmath,color,graphicx, multicol, nicefrac}
\usepackage[inline]{asymptote}

\nofiles

\setlength{\parindent}{0in}

\pagestyle{plain}

\title{Quiz template}
\author{Wray}
\begin{document}

\begin{asydef}
//
// Global Asymptote definitions can be put here.
//
usepackage("bm");
texpreamble("\def\V#1{\bm{#1}}");
\end{asydef}



\noindent {\sc {\bf {\Large Retake 6A: Graphs and geometry}}
            \hfill Math 1110, Fall 2017}
\bigskip

\noindent {\sc  {\large Name:}
             \hfill}
             
\bigskip
\bigskip

\smallskip

Graph each of the functions.  Your graph should show the correct shape and the coordinates of the two or three  points that are easy to find.

\begin{questions}

\begin{multicols}{2}
\question[3]
$f(x) = x^{3}$

\vspace{\stretch{1}}

\begin{asy}
size(7cm);
import graph;
for (int i = -5; i <= 5; ++i)
	{
    draw((-5,i)--(5,i), gray);
    draw((i,-5)--(i,5), gray);
    }
draw(Label("$x$", EndPoint, E), (-6,0)--(6,0), Arrows);
draw(Label("$y$", EndPoint, N), (0,-6)--(0,6), Arrows);
\end{asy}

\vspace{\stretch{1}}

\question[3]
$f(x) = (x+2)^{3}$

\vspace{\stretch{1}}

\begin{asy}
size(7cm);
import graph;
for (int i = -5; i <= 5; ++i)
	{
    draw((-5,i)--(5,i), gray);
    draw((i,-5)--(i,5), gray);
    }
draw(Label("$x$", EndPoint, E), (-6,0)--(6,0), Arrows);
draw(Label("$y$", EndPoint, N), (0,-6)--(0,6), Arrows);
\end{asy}

\vspace{\stretch{1}}

\end{multicols} 

\vspace{\stretch{2}}

\begin{multicols}{2}
\question[3]
$f(x) = x^{\nicefrac{1}{4}}$

\vspace{\stretch{1}}

\begin{asy}
size(7cm);
import graph;
for (int i = -5; i <= 5; ++i)
	{
    draw((-5,i)--(5,i), gray);
    draw((i,-5)--(i,5), gray);
    }
draw(Label("$x$", EndPoint, E), (-6,0)--(6,0), Arrows);
draw(Label("$y$", EndPoint, N), (0,-6)--(0,6), Arrows);
\end{asy}

\vspace{\stretch{1}}

\question[3]
$f(x) = - x^{-2} - 3$

\vspace{\stretch{1}}

\begin{asy}
size(7cm);
import graph;
for (int i = -5; i <= 5; ++i)
	{
    draw((-5,i)--(5,i), gray);
    draw((i,-5)--(i,5), gray);
    }
draw(Label("$x$", EndPoint, E), (-6,0)--(6,0), Arrows);
draw(Label("$y$", EndPoint, N), (0,-6)--(0,6), Arrows);
\end{asy}

\vspace{\stretch{1}}

\end{multicols} 

\vspace{\stretch{3}}
\newpage 

\question[8]
A garden is constructed in the shape of a right triangle.  The legs of the triangle have lengths 20 feet and $x$ feet.  Find a function that expresses the area of the garden as a function of $x$. \newline
\bigskip
\bigskip
\vspace{\stretch{1}}
\begin{asy}
size(5cm);
draw((0,0) -- (0,7), white);
draw(Label("$x$", MidPoint, 2*SW + W), (0,0)--(0,5), linewidth(1.5bp));
draw(Label("$20$", MidPoint, 2*S), (0,0)--(2*sqrt(6),0), linewidth(1.5bp));
draw( (0,5)--(2*sqrt(6),0), linewidth(1.5bp));

draw(box((0,0), (0.5,0.5)));
\end{asy}

\vspace{\stretch{8}}

\end{questions}

\end{document}