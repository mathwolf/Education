\documentclass[addpoints, 12pt]{exam}

\setlength{\topmargin}{-.75in} \addtolength{\textheight}{2.00in}
\setlength{\oddsidemargin}{.00in} \addtolength{\textwidth}{.25in}

\usepackage{amsmath,color,graphicx, multicol, nicefrac}
\usepackage[inline]{asymptote}

\nofiles

\setlength{\parindent}{0in}

\pagestyle{plain}

\title{Quiz template}
\author{Wray}
\begin{document}

\begin{asydef}
//
// Global Asymptote definitions can be put here.
//
usepackage("bm");
texpreamble("\def\V#1{\bm{#1}}");
\end{asydef}



\noindent {\sc {\bf {\Large Retake 7A: Polynomials}}
            \hfill Math 1110, Fall 2017}
\bigskip

\noindent {\sc  {\large Name:}
             \hfill}
             
\bigskip
\bigskip

\begin{questions}

\question
Find $\left( x^3 - 1 \right) \div \left( x + 1 \right)$.  Remember to include the remainder term in your answer, if needed.

\clearpage

\question
Consider the polynomial function.
\begin{align*}
f(x) = - x^4 + x^3 + x^2 - x
\end{align*}

\begin{parts}
\part 
Give the degree and the leading term.

\vspace{\stretch{2}}

\part 
Factor the polynomial and give all its zeros.  In addition, give the multiplicity for each zero that you identify.

\vspace{\stretch{6}}

\part 
Calculate the coordinates of one test point in between each pair of zeros you found on part (b).

\vspace{\stretch{3}}

\part 
Graph the polynomial.

\vspace{\stretch{1}}
\begin{center}
\begin{asy}
size(8cm);
for (int i = -10; i <= 10; ++i)
	{
    draw((-10,i)--(10,i), mediumgray);
    draw((i,-10)--(i,10), mediumgray);
    }
\end{asy}
\end{center}
\vspace{\stretch{1}}

\end{parts}

\end{questions}

\end{document}