\documentclass[addpoints, 12pt]{exam}

\setlength{\topmargin}{-.75in} \addtolength{\textheight}{2.00in}
\setlength{\oddsidemargin}{.00in} \addtolength{\textwidth}{.25in}

\usepackage{amsmath,color,graphicx, multicol, nicefrac}
\usepackage[inline]{asymptote}

\nofiles

\setlength{\parindent}{0in}

\pagestyle{plain}

\title{Quiz template}
\author{Wray}
\begin{document}

\begin{asydef}
//
// Global Asymptote definitions can be put here.
//
usepackage("bm");
texpreamble("\def\V#1{\bm{#1}}");
\end{asydef}



\noindent {\sc {\bf {\Large Quiz 5A: Algebra of functions}}
            \hfill Math 1110, Fall 2017}
\bigskip

\noindent {\sc  {\large Name:}
             \hfill}
             
\bigskip
\bigskip

\smallskip

\begin{questions}

\question[6]
Define the function $f(x) = 6 - 2x^2$.  Simplify the following difference quotient.
\begin{align*}
\dfrac{f(x+h) - f(x)}{h}
\end{align*}

\vspace{\stretch{3}}

\question[8]
Write the formula for the piecewise function $g(x)$ shown in the graph.

\vspace{\stretch{1}}

\begin{asy}
size(7cm);
import graph;
for (int i = -5; i <= 5; ++i)
	{
    draw((-5,i)--(5,i), gray);
    draw((i,-5)--(i,5), gray);
    }
draw(Label("$x$", EndPoint, E), (-6,0)--(6,0), Arrows);
draw(Label("$g(x)$", EndPoint, N), (0,-6)--(0,6), Arrows);

draw( (1,3)--(5,3), Arrow);
draw( (1,1)--(-5,-2), Arrow);
dotfactor *= 2;
dot((1,3)); 
dot((1, 1), filltype=FillDraw(fillpen=white, drawpen=black));

\end{asy}

\vspace{\stretch{2}}

\newpage 

\question[6]
The following graph shows a function $f(x)$.  Estimate the rate of change for this function over the interval $x = -1$ to $x = 2$.

\vspace{\stretch{1}}
\begin{center}
\begin{asy}
size(8cm);
import graph;
for (int i = -5; i <= 5; ++i)
	{
    draw((-5,i)--(5,i), gray);
    draw((i,-5)--(i,5), gray);
    }
draw(Label("$x$", EndPoint, E), (-6,0)--(6,0), Arrows);
draw(Label("$y$", EndPoint, N), (0,-6)--(0,6), Arrows);
pair F(real t) { 
	return (t, (27.0/64.0) * (t + 5.6)^2 * ( (1.0/8.0) * (t + 5.6) - 1) + 2.1);
}
path g = graph(F, -4.7, 3.1, operator ..);
draw(g);

\end{asy}
\end{center}

\vspace{\stretch{15}}

\end{questions}

\end{document}