\documentclass[addpoints, 12pt]{exam}

\setlength{\topmargin}{-.75in} \addtolength{\textheight}{2.00in}
\setlength{\oddsidemargin}{.00in} \addtolength{\textwidth}{.25in}

\usepackage{amsmath,color,graphicx, multicol}
\usepackage[inline]{asymptote}

\nofiles

\setlength{\parindent}{0in}

\pagestyle{plain}

\title{Quiz template}
\author{Wray}
\begin{document}

\begin{asydef}
//
// Global Asymptote definitions can be put here.
//
usepackage("bm");
texpreamble("\def\V#1{\bm{#1}}");
\end{asydef}



\noindent {\sc {\bf {\Large Quiz 4B: Symmetry and functions}}
            \hfill Math 1110, Fall 2017}
\bigskip

\noindent {\sc  {\large Name:}
             \hfill}
             
\bigskip
\bigskip

\smallskip

\begin{questions}

\question[5]
Part of a function $f(x)$ is shown in the following graph.  Suppose you know this function is symmetric with respect to the origin.  Use this information to sketch the remaining part of the graph.

\vspace{\stretch{1}}

\begin{asy}
size(8cm);
import graph;
for (int i = -5; i <= 5; ++i)
	{
    draw((-5,i)--(5,i), gray);
    draw((i,-5)--(i,5), gray);
    }
draw(Label("$x$", EndPoint, E), (-6,0)--(6,0), Arrows);
draw(Label("$y$", EndPoint, N), (0,-6)--(0,6), Arrows);
pair F(real t) { 
	return (2.5 * atan(t), t);
}
path g = graph(F, 0, 4.5, operator ..);
draw(g);

\end{asy}

\vspace{\stretch{1}}

\question[5]
Consider the following equation.
\begin{align*}
y^2 = x^3 - 2
\end{align*}
Test this equation to see if the solutions are symmetric with respect to the $x$-axis, the $y$-axis, or the origin.

\vspace{\stretch{3}}

\newpage

\question[10]
Define the function $h(x) = x^2 - 3x$. Evaluate each expression, and simplify your answer as much as possible.

\begin{parts}

\part
$h(-2)$
\vspace{\stretch{2}}

\part
$h(0)$
\vspace{\stretch{2}}

\part
$h(b-1)$
\vspace{\stretch{3}}

\end{parts}

\end{questions}

\end{document}