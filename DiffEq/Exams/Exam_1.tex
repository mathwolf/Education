\documentclass[12pt]{exam}

\setlength{\topmargin}{-.75in} \addtolength{\textheight}{2.00in}
\setlength{\oddsidemargin}{.00in} \addtolength{\textwidth}{.25in}

\usepackage{amsmath,color,graphicx, multicol, nicefrac, enumerate}
\usepackage[inline]{asymptote}

\nofiles

\setlength{\parindent}{0in}

\pagestyle{plain}

\title{Differential equations exam 1 - SP 17}
\author{Wray}
\begin{document}

\begin{asydef}
//
// Global Asymptote definitions can be put here.
//
usepackage("bm");
texpreamble("\def\V#1{\bm{#1}}");
\end{asydef}

\noindent {\sc {\bf {\Large Exam 1}}
            \hfill MAT 265, Spring 2017}
\bigskip
\bigskip

\begin{questions}

\question[10]
Recall that acceleration is defined as the rate of change in velocity.  Suppose that the acceleration of a motorboat is proportional to the difference between 50 km/h and the boat's velocity.  Write a differential equation that models this situation.  Then find the general solution of your equation.

\vspace{\stretch{4}}

\question[10]
Use the theorem from class to discuss the existence and uniqueness of solutions to the following IVPs.
\begin{parts}
\part
$\dfrac{dy}{dx} = \sqrt{1 - y}, \> y(0) = 2$

\vspace{\stretch{2}}

\part
$\dfrac{dy}{dx} = \sqrt{1 - x}, \> y(0) = 2$
\end{parts}

\vspace{\stretch{3}}

\clearpage

\question
Solve the differential equation or IVP.
\begin{parts}
\part[12]
$\left( 3x^2 + 2y^2 \right) dx + \left( 4xy + 6y^2 \right) dy = 0$

\clearpage

\part[13]
$y^{\prime} + \dfrac{3}{t} y = \dfrac{\sin t}{t^2}, \> t>0, \> y(1) = \sin 1$


\clearpage

\part[20]
$ty^{\prime} + y = t \ln \left( t \right) y^2, \> t>0, \> y(1) = \dfrac{1}{2}$ 

\end{parts}

\clearpage

\question
In class we discussed the logistic model for population growth.  We could also add the assumption that a minimum size is necessary to maintain a healthy population.  This leads to the revised model
\begin{align*}
\dfrac{dP}{dt} = k P \left( 1 - \dfrac{P}{M} \right) \left( 1 - \dfrac{m}{P} \right).
\end{align*}
We consider a population with parameters $k = \dfrac{1}{4}, M = 100,$ and $m = 20$.

\begin{parts}
\part[5]
Show that the equation with these parameters is equivalent to 
\begin{align*}
\dfrac{dP}{dt} = \dfrac{1}{400} \left( 100 - P \right) \left( P - 20 \right).
\end{align*}

\vspace{\stretch{1}}

\part[10]
Draw the phase portrait for this equation and describe the stability of each critical point.

\vspace{\stretch{3}}

\part[5]
Make a graph that shows the equilibrium solutions that you identified in part (b).  On the same graph, sketch solution curves for the initial conditions $y(0) = 15, \> y(0) = 45, \> y(0) = 75,$ and $y(0) = 115$.

\vspace{\stretch{3}}

\part[5]
For what initial values will the population go extinct?

\vspace{\stretch{2}}

\clearpage

\part[10]
Solve the IVP
\begin{align*}
	\dfrac{dP}{dt} = \dfrac{1}{400} \left( 100 - P \right) \left( P - 20 \right), \> P(0) = 50.
\end{align*}
You can leave your solution in implicit form.

\end{parts}

\end{questions}
\end{document}