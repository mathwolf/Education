\documentclass[12pt]{exam}

\setlength{\topmargin}{-.75in} \addtolength{\textheight}{2.00in}
\setlength{\oddsidemargin}{.00in} \addtolength{\textwidth}{.25in}

\usepackage{amsmath,color,graphicx, multicol, nicefrac, enumerate}
\usepackage[inline]{asymptote}

\nofiles

\setlength{\parindent}{0in}

\pagestyle{plain}

\title{Differential equations final exam review - SP 17}
\author{Wray}
\begin{document}

\begin{asydef}
//
// Global Asymptote definitions can be put here.
//
usepackage("bm");
texpreamble("\def\V#1{\bm{#1}}");
\end{asydef}

\noindent {\sc {\bf {\Large Final exam review}}
            \hfill MAT 265, Spring 2017}
\newline
\newline

\subsection*{First order problems}

You should be able to use the following techniques to solve a first order equation. \\

Separation of variables \\
Integrating factors \\
Substitutions, including the Bernoulli method \\

Also be prepared to write the differential equation that corresponds to a written description.  The specific applications that we worked with were population growth, radioactive decay, and mixtures of solutions.  You should also be able to use a phase line to discuss the solutions of an autonomous problem.

\subsection*{Second order problems}

Use the following techniques to solve a second order problem.\\

Method of undetermined coefficients \\
Variation of parameters \\

You will need to set up an equation for a mass-spring system.  This includes understanding how the terms of the differential equation corresponds to the parameters of the physical system.  

\subsection*{Laplace transforms and power series methods}

Find the transform of a simple function using the definition, and use a table of transforms to solve a more complicate problem.  Understand how to solve equations using unit step functions or periodic functions.  You will also need to incorporate an impulse (like a hammer strike) into a mass-spring model. \\

Use the power series method to solve a differential equation with variable coefficients, similar to the question on Exam 3.

\subsection*{Systems of differential equations}

Use the eigenvalue method to solve a system of differential equations.  You should be prepared for problems with real distinct eigenvalues and problems with complex conjugate eigenvalues. \\

Here are some practice problems on the last topic.

\begin{questions}

\question
$\dfrac{dx}{dt} = -4x + 2y \smallskip \\ \dfrac{dy}{dt} = 2x - 4y$

\clearpage

\question
$\dfrac{dx}{dt} = x + 2y \smallskip \\ \dfrac{dy}{dt} = -2x + y$

\question
$\dfrac{dx}{dt} = 3x \smallskip \\ \dfrac{dy}{dt} = 6x  -3y \smallskip \\ x(0) = 2, \> y(0) = -5$

\question
$\dfrac{dx}{dt} = 4x + 5y \smallskip \\ \dfrac{dy}{dt} = -5x - 4y \smallskip \\ x(0) = 0, \> y(0) = 3$

\end{questions}

\bigskip
SOLUTIONS

\begin{questions}
\question
$x(t) = C_1e^{-6t} + C_2 e^{-2t} \\
y(t) = -C_1e^{-6t} + C_2 e^{-2t}$

\question
$x(t) = C_1 e^t \sin 2t - C_2 e^t \cos 2t \\
y(t) = C_1 e^t \cos 2t + C_2 e^t \sin 2t$

\question
$x(t) = 2 e^{3t} \\
y(t) = -7 e^{-3t} + 2 e^{3t}$

\question
$x(t) = 5 \sin 3t \\
y(t) = 3 \cos 3t - 4 \sin 3t$

\end{questions}
\end{document}