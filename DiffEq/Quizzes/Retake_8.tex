\documentclass[addpoints, 12pt]{exam}

\setlength{\topmargin}{-.75in} \addtolength{\textheight}{2.00in}
\setlength{\oddsidemargin}{.00in} \addtolength{\textwidth}{.25in}

\usepackage{amsmath,color,graphicx, multicol, nicefrac}
\usepackage[inline]{asymptote}

\newcommand{\laplace}[1]{\mathcal{L} \left\lbrace #1 \right\rbrace}

\setlength{\parindent}{0in}

\pagestyle{plain}

\title{Quiz template}
\author{Wray}
\begin{document}

\begin{asydef}
//
// Global Asymptote definitions can be put here.
//
usepackage("bm");
texpreamble("\def\V#1{\bm{#1}}");
\end{asydef}



\noindent {\sc {\bf {\Large Retake 8}}
            \hfill MAT 265, Spring 2017}
\bigskip

\noindent {\sc  {\large Name:}
             \hfill}
             
\bigskip
\bigskip

\smallskip

\begin{questions}

\question[20]
Recall that the Laplace transform of a function is
\begin{align*}
	\laplace{f(t)} = \int_0^{\infty} e^{-st} f(t) \> dt.
\end{align*}
Use the definition to find the transform of $f(t) = t^2 + e^{-t}$.
\clearpage

\end{questions}

\end{document}