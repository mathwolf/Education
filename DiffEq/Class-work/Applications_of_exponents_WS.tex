\documentclass[12pt]{article}

\title{Worksheet template}

\setlength{\topmargin}{-.75in} \addtolength{\textheight}{2.00in}
\setlength{\oddsidemargin}{.00in} \addtolength{\textwidth}{.75in}

\usepackage{amsmath,color,graphicx}
\usepackage[inline]{asymptote}
\usepackage{gensymb, multicol, verbatim, enumitem, nicefrac}

\nofiles

\pagestyle{empty}

\setlength{\parindent}{0in}


\begin{document}

\noindent {\sc {\bf {\Large Applications of exponents and logs}}}
\bigskip
\bigskip


\begin{enumerate}

\item
The half-life of Zn-71 is 2.4 minutes.  It there was 100.0 g at the beginning of an experiment, how many grams remain after 7.25 minutes has elapsed?

\item Pd-100 has a half-life of 3.6 days.  If you have $6.02 \times 10^{23}$ atoms initially, how many atoms will be present after 20.0 days?

\item
Os-182 has a half-life of 21.5 hours.  How many grams of a 10.0 gram sample would have decayed after exactly three half-lives?

\item
After 24.0 days, 2.00 mg remains from a sample that originally contained 128.0 mg.  What is the half-life of the element in the sample?

\item
U-238 has a half-life of $4.46 \times 10^9$ years.  How much U-238 should be present in a sample that is $2.5 \times 10^9$ years old if there was 2.00 grams present initially

\item
Fe-253 has a half-life of 0.334 seconds.  A radioactive sample is considered to be completely decayed after 10 half-lives.  How much time will pass until a sample of this element is completely decayed?

\end{enumerate}

\clearpage

SOLUTIONS

\begin{enumerate}

\item 12.5 g
\item $1.28 \times 10^{22} atoms$
\item 8.75 g
\item 4.00 days
\item 1.36 g
\item 3.34 s
\end{enumerate}


\end{document}
