\documentclass[addpoints, 12pt]{exam}

\setlength{\topmargin}{-.75in} \addtolength{\textheight}{2.00in}
\setlength{\oddsidemargin}{.00in} \addtolength{\textwidth}{.25in}

\usepackage{amsmath,color,graphicx, multicol}
\usepackage[inline]{asymptote}

\nofiles

\setlength{\parindent}{0in}

\pagestyle{plain}

\title{Worksheet template}
\author{Wray}
\begin{document}

\begin{asydef}
//
// Global Asymptote definitions can be put here.
//
usepackage("bm");
texpreamble("\def\V#1{\bm{#1}}");
\end{asydef}

\smallskip
\begin{questions}

\question
Consider the initial value problem
\begin{align*}
	y^{\prime} \ + y = x + 1, \> y(0) = 0
\end{align*}
Is this problem guaranteed to have a unique solution?

\bigskip

\question
Consider the initial value problem
\begin{align*}
	y^{\prime} x = 2y, \> y(0) = 1.
\end{align*}
Is this problem guaranteed to have a unique solution?

\bigskip

\question
Consider the initial value problem
\begin{align*}
	y^{\prime} - y^2 - 1 = 0, \> y(0) = 0
\end{align*}
Is this problem guaranteed to have a unique solution?

\bigskip

\question
Consider the initial value problem
\begin{align*}
	y^{\prime} = 2 \sqrt{ \lvert y \rvert }, \> y(0) = 0.
\end{align*}

\begin{parts}
\part
Show that the function $y(x) = 0$ is a solution to the IVP.

\part
Show that the function $y(x) = x \lvert x \rvert$ is a solution to the IVP.

\part
The previous two parts show that this problem does not have a unique solution.  (It has two.)  Explain why this result is consistent with the existence and uniqueness theorem that we discussed in class.
\end{parts}

\end{questions}
\end{document}