\documentclass[addpoints, 12pt]{exam}

\setlength{\topmargin}{-.75in} \addtolength{\textheight}{2.00in}
\setlength{\oddsidemargin}{.00in} \addtolength{\textwidth}{.25in}

\usepackage{amsmath,color,graphicx, multicol, enumitem}
\usepackage[inline]{asymptote}

\nofiles

\setlength{\parindent}{0in}

\pagestyle{plain}

\title{Diff Eq honors project}
\author{Wray}
\begin{document}

\begin{asydef}
//
// Global Asymptote definitions can be put here.
//
usepackage("bm");
texpreamble("\def\V#1{\bm{#1}}");
\end{asydef}


\noindent {\sc {\bf {\Large Differential equations honors project}}
            \hfill MAT 265, Spring 2017}
\bigskip

\bigskip

\textbf{Drag force} \bigskip
\newline

One model for the magnitude of the force due to air resistance is
\begin{align*}
	W = \dfrac{1}{2} c_W \rho A v^2,
\end{align*}
where $c_W$ is a drag coefficient, $\rho$ is the density of air, $A$ is the 
cross-sectional area of the projectile, and $v$ is its velocity.  Eventually we will put
in numbers for the parameters in this model.  For now, just use $kv^2$ to describe this force. \bigskip

Air resistance always points opposite the direction of an object's motion.
\bigskip

\textbf{One dimensional motion} \bigskip
\newline
In the simplest model, we will assume that the potato is moving in just the vertical direction.  At time 
$t=0$, the potato is launched upward with an initial speed $v_0$.

\begin{questions}

\question
Draw a diagram showing the forces on the potato while it is traveling upward.  Use $W$ to denote the drag force.  Make sure to label the positive direction.

\question
Use the description of $W$ above together with Newton's second law to write a differential equation for $v$ that corresponds to your diagram.  One side of the equation will have $\dfrac{dv}{dt}$.

\question
Solve the problem to obtain $t^{\star}$, the time when the projectile reaches its maximum height.  You should use a definite integral on both sides of your equation instead of adding a 
$+C$.  See me on Thursday after class if you have questions about setting up this step.

\question
Repeat the above steps to find a function for the velocity on the trip downward.

\end{questions}
\end{document}