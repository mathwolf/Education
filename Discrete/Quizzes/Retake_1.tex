\documentclass[addpoints, 12pt]{exam}

\setlength{\topmargin}{-.75in} \addtolength{\textheight}{2.00in}
\setlength{\oddsidemargin}{.00in} \addtolength{\textwidth}{.25in}

\usepackage{amsmath,color,graphicx, multicol}
\usepackage[inline]{asymptote}

\nofiles

\setlength{\parindent}{0in}

\pagestyle{plain}

\title{Quiz template}
\author{Wray}
\begin{document}

\begin{asydef}
//
// Global Asymptote definitions can be put here.
//
usepackage("bm");
texpreamble("\def\V#1{\bm{#1}}");
\end{asydef}



\noindent {\sc {\bf {\Large Retake 1: Linear equations}}
            \hfill MAT 123, Summer 2016}
\bigskip

\noindent {\sc  {\large Name:}
             \hfill}
             
\bigskip
\bigskip

\smallskip

\begin{questions}

\question[5]
Solve $3x + 25 = 5x$.

\vspace{\stretch{3}}

\question[5]

Solve $B = \dfrac{1}{2} \left( h + a \right)$ for $a$.

\vspace{\stretch{5}}

\newpage

\question[5]

Find the slope of the line passing through the points $(-3,3)$ and $(3,-3)$.

\vspace{\stretch{3}}

\question[5]
Graph the line $3x - 4y = -16$.

\vspace{\stretch{1}}

\begin{asy}
size(8cm);
for (int i = -5; i <= 5; ++i)
	{
    draw((-5,i)--(5,i), mediumgray);
    draw((i,-5)--(i,5), mediumgray);
    }
draw(Label("$x$", EndPoint, E), (-6,0)--(6,0), Arrows);
draw(Label("$y$", EndPoint, N), (0,-6)--(0,6), Arrows);

\end{asy}

\vspace{\stretch{1}}

\end{questions}

\end{document}