\documentclass[12pt]{exam}

\setlength{\topmargin}{-.75in} \addtolength{\textheight}{2.00in}
\setlength{\oddsidemargin}{.00in} \addtolength{\textwidth}{.25in}

\usepackage{amsmath,color,graphicx, multicol, nicefrac}
\usepackage[inline]{asymptote}

\nofiles

\setlength{\parindent}{0in}

\pagestyle{plain}

\title{Conditional probability}
\author{Wray}
\begin{document}

\begin{asydef}
//
// Global Asymptote definitions can be put here.
//
usepackage("bm");
texpreamble("\def\V#1{\bm{#1}}");
\end{asydef}

\noindent {\sc {\bf {\Large Conditional probability}}
            \hfill MAT 123, Summer 2016}
\bigskip
\bigskip

We use the formula for conditional probability in two ways.
\begin{gather*}
P(A \> \vert B) = \dfrac{P(A \cap B)}{P(B)} \\
P(A \cap B) = P(B) P(A \> \vert B)
\end{gather*}

\begin{questions}
\question
We have a jar with 5 blue chips and 6 yellow chips.  We draw two chips from the jar.  
\begin{parts}
\part
What is the probability that the first chip is blue?  What is the probability that the first chip is yellow?

\part
If the first chip drawn is blue, what is the probability that the second chip is also blue?

\part
If the first chip is yellow, what is the probability that the second chip is blue?

\part
What is the probability of drawing two blue chips?

\part
What is the probability of drawing a yellow chip first and then a blue chip?

\part
What is the probability that the second chip is blue?

\part
What is the probability that the second chip is yellow? 

\end{parts}

\question
We roll two dice.
\begin{parts}
\part
What is the probability that the first die is a six?

\part
What is the probability that the sum of the two dice is greater than or equal to 10?

\part
If the first die is a six, what is the probability that the sum of the two dice is greater than or equal to 10?

\part
If the sum of the two dice is greater than 10, what is the probability that the first die is a six?

\part
What is the probability that the first die is six and the sum of the two dice is greater than or equal to 10?

\end{parts}

\question
We flip a coin.  If the coin shows heads, we roll a die.  If the coin shows tails, we turn the die so that it reads 6.
\begin{parts}
\part
What is the probability of getting heads on the coin?

\part
If the coin is heads, what is the probability of getting a 6 on the die?

\part
If the coin shows tails, what is the probability of getting a 6?

\part
What is the probability of getting heads and rolling a 6?

\part
What is the probability of getting tails and rolling a 6?

\part
What is the probability of getting a 6?

\end{parts}

\clearpage
\question
Two factories make parts for an automaker.  A part is equally likely to come from either of the factories.  Factory A has a defect rate of 1\%, and factory B has a defect rate of 4\%. 
\begin{parts}
\part
What is the probability that a part was made in factory A?

\part
If a part was made in factory A, what is the probability that it is defective?

\part
If a part was made in factory B, what is the probability that it is defective?

\part
What is the probability that a part is defective and is made in factory A?

\part
What is the probability that a part is defective and is made in factory B?

\part
What is the probability that a part is defective?

\part
What is the probability that a part is not defective?

\end{parts}

\question
An urn contains 5 blue chips, 3 white chips, and 7 red chips.  Two chips are drawn.

\begin{parts}
\part
What is the probability that the first chip is blue?  White?  Red?

\part
If the first chip is blue, what is the probability that the second chip is blue?

\part
If the first chip is blue, what is the probability that the second chip is not blue?

\part
What is the probability that both chips are blue?

\part
What is the probability that both chips are the same color?

\part
What is the probability that the two chips are different colors?
\end{parts}

\end{questions}

\bigskip
\bigskip
ANSWERS
\begin{multicols}{3}
\begin{questions}
\question 
\begin{parts}
\part $\nicefrac{5}{11}$, $\nicefrac{6}{11}$
\part $\nicefrac{2}{5}$
\part $\nicefrac{1}{2}$
\part $\nicefrac{2}{11}$
\part $\nicefrac{3}{11}$
\part $\nicefrac{5}{11}$
\part $\nicefrac{6}{11}$
\end{parts}

\question 
\begin{parts}
\part $\nicefrac{1}{6}$
\part $\nicefrac{1}{6}$
\part $\nicefrac{1}{2}$
\part $\nicefrac{1}{2}$
\part $\nicefrac{1}{12}$
\end{parts}

\question 
\begin{parts}
\part $\nicefrac{1}{2}$
\part $\nicefrac{1}{6}$
\part 1
\part $\nicefrac{1}{12}$
\part $\nicefrac{1}{2}$
\part $\nicefrac{7}{12}$
\end{parts}

\question 
\begin{parts}
\part $\nicefrac{1}{2}$
\part 1\% or $\nicefrac{1}{100}$
\part 4\% or $\nicefrac{1}{25}$
\part $\nicefrac{1}{200}$
\part $\nicefrac{1}{50}$
\part $\nicefrac{1}{40}$
\part $\nicefrac{39}{40}$
\end{parts}

\question 
\begin{parts}
\part $\nicefrac{1}{3}$, $\nicefrac{1}{5}$, $\nicefrac{7}{15}$
\part $\nicefrac{2}{7}$
\part $\nicefrac{5}{7}$
\part $\nicefrac{2}{21}$
\part $\nicefrac{34}{105}$
\part $\nicefrac{71}{105}$
\end{parts}

\end{questions}
\end{multicols}

\end{document}