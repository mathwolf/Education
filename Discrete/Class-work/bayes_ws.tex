\documentclass[12pt]{exam}

\setlength{\topmargin}{-.75in} \addtolength{\textheight}{2.00in}
\setlength{\oddsidemargin}{.00in} \addtolength{\textwidth}{.25in}

\usepackage{amsmath,color,graphicx, multicol, nicefrac}
\usepackage[inline]{asymptote}

\nofiles

\setlength{\parindent}{0in}

\pagestyle{plain}

\title{Bayes' formula}
\author{Wray}
\begin{document}

\begin{asydef}
//
// Global Asymptote definitions can be put here.
//
usepackage("bm");
texpreamble("\def\V#1{\bm{#1}}");
\end{asydef}

\noindent {\sc {\bf {\Large Bayes' formula}}
            \hfill MAT 123, Summer 2016}
\bigskip
\bigskip

To use Bayes' formula, we start with the definition of conditional probability.
\begin{gather*}
P(B \> \vert A) = \dfrac{P(A \cap B)}{P(A)}
\end{gather*}
Then we expand each part of the fraction, again using conditional probability.
\begin{gather*}
P(A \cap B) = P(A \> \vert B) P(B) \\
P(A) = P(A \> \vert B) P(B) + P(A \> \vert B^{\prime}) P(B^{\prime})
\end{gather*}

\begin{questions}
\question
We have a jar with 5 blue chips and 6 yellow chips.  We draw two chips from the jar.  
\begin{parts}
\part
If the second chip is blue, what is the probability that the first chip was blue?

\part
If the second chip is blue, what is the probability that the first chip was yellow?

\part
If the second chip is yellow, what is the probability that the first chip was blue?

\end{parts}

\question
We flip a coin.  If the coin shows heads, we roll a die.  If the coin shows tails, we turn the die so that it reads 6.
\begin{parts}
\part
If the die shows 6, what is the probability that we flipped a tail?

\part
If the die shows 6, what is the probability that we flipped a head?

\part
If the die shows 3, what is the probability that we flipped a tail?

\end{parts}

\question Two factories make parts for an automaker.  A part is equally likely to come from either of the factories.  Factory A has a defect rate of 1\%, and factory B has a defect rate of 4\%. If a part is defective, what is the probability that it was made in factory B?

\question
An urn contains 5 blue chips, 7 red chips, and 3 white chips.  Two chips are drawn.

\begin{parts}
\part
If the second chip is blue, what is the probability that the first chip was blue?

\part
If the second chip is blue, what is the probability that the two chips are different colors?

\end{parts}

\end{questions}

\bigskip
\bigskip
ANSWERS
\begin{multicols}{3}
\begin{questions}
\question 
\begin{parts}
\part $\nicefrac{2}{5}$
\part $\nicefrac{3}{5}$
\part $\nicefrac{1}{2}$
\end{parts}

\question 
\begin{parts}
\part $\nicefrac{6}{7}$
\part $\nicefrac{1}{7}$
\part 0
\end{parts}

\question 
$\nicefrac{4}{5}$ = 80 \%

\question 
\begin{parts}
\part $\nicefrac{12}{21}$
\part $\nicefrac{9}{21}$
\end{parts}

\end{questions}
\end{multicols}
\end{document}