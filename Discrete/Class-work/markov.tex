\documentclass[12pt]{exam}

\setlength{\topmargin}{-.75in} \addtolength{\textheight}{2.00in}
\setlength{\oddsidemargin}{.00in} \addtolength{\textwidth}{.25in}

\usepackage{amsmath,color,graphicx, multicol, nicefrac}
\usepackage[inline]{asymptote}

\nofiles

\setlength{\parindent}{0in}

\pagestyle{plain}

\title{Markov chains}
\author{Wray}
\begin{document}

\begin{asydef}
//
// Global Asymptote definitions can be put here.
//
usepackage("bm");
texpreamble("\def\V#1{\bm{#1}}");
\end{asydef}

\noindent {\sc {\bf {\Large Markov chains}}
            \hfill MAT 123, Summer 2016}
\bigskip
\bigskip

\begin{questions}
\question
A rental car company has one location at DIA and a second location downtown.  It is possible for a customer to rent a car in location and return it to the other.  If a car is rented at DIA, the probability that it will be returned to DIA is 0.8.  If a car is rented downtown, the probability that it will be returned downtown is 0.7.  For our first simple model, we will assume that a car is rented and returned once per day.
\begin{parts}
\part
Draw a transition diagram for the state of a rental car.

\part
Make the transition matrix.

\part
Suppose that the company begins with half of their cars at DIA and half downtown.  Write the state matrix corresponding to this setup.  

\part
Find the next state matrix in the Markov chain.  What percentage of the cars will be at DIA on the next day?

\part
Find the percentage of cars at DIA on day 2 and on day 10.

\part
Is the transition matrix for this system regular?  If so, describe the way the rental fleet will be split between locations in the long term.
\end{parts}

\question
Based on census data, economists estimate that 95\% of people who on their own home today will also own their home in four years.  On the other hand, 15\% off people who do not own their own home today will own a home in four years.  

\begin{parts}
\part
Draw a transition diagram for the state of a customer's preference.

\part
Make the transition matrix.

\part
Suppose that currently 65.4\% of people own their own home.  Write the initial state matrix for this data.

\part
Find the next state matrix in the Markov chain.  What percentage of people will own their home in four years?

\part
Find the percentage of homeowners in 20 years.

\part
Is the transition matrix for this system regular?  If so, what percentage of people will own their own home in the long run?
\end{parts}

\question
Customers in a certain town can choose between McDonald's, Chipotle, and Pizza Hut.  Every day, McDonald's loses 10\% of its customers to Chipotle and 20\% to Pizza Hut; Chipotle loses 15\% of its customers to McDonald's and 10\% to Pizza Hut; and Pizza Hut loses 5\% of its customers to McDonald's and 5\% to Pizza Hut.  
\begin{parts}
\part
Draw a transition diagram for a consumer's preference.

\part
Make the transition matrix.

\part
Suppose that at the beginning of our study 40\% of consumers prefer McDonald's, 30\% prefer Chipotle, and 30\% prefer Pizza Hut.  Write the state matrix for this data.

\part
Find the next state matrix in the Markov chain, and describe the corresponding consumer preferences.

\part
How will customers be split between the restaurants in the long term?
\end{parts}

\end{questions}

\clearpage
ANSWERS
\begin{questions}
\begin{multicols}{2}
\question 
\begin{parts}
\setcounter{partno}{1}
\part
$\left[ \begin{array}{cc}
0.8 & 0.2 \\ 0.3 & 0.7 
\end{array}
\right]$

\part
$\left[ \begin{array}{cc}
0.5 & 0.5
\end{array}
\right]$

\part
$\left[ \begin{array}{cc}
0.55 & 0.45
\end{array} 
\right]$

55\%

\part
57.5\%

60\%

\part
The long-term split has 60\% of the cars at DIA and 40\% downtown.
\end{parts}
\columnbreak
\question 
\begin{parts}
\setcounter{partno}{1}
\part
$\left[ \begin{array}{cc}
0.95 & 0.05 \\ 0.15 & 0.85 
\end{array}
\right]$

\part
$\left[ \begin{array}{cc}
0.654 & 0.346
\end{array}
\right]$

\part
$\left[ \begin{array}{cc}
0.6886 & 0.3114
\end{array} 
\right]$

68.9\%

\part
71.9\%

\part
75\%
\end{parts}
\end{multicols}

\bigskip
\question 
\begin{parts}
\setcounter{partno}{1}
\part
$\left[ \begin{array}{ccc}
0.7 & 0.1 & .2 \\ 0.15 & 0.75 & 0.1 \\ 0.05 & 0.05 & 0.9 
\end{array}
\right]$

\part
$\left[ \begin{array}{ccc}
0.4 & 0.3 & 0.3
\end{array}
\right]$

\part
$\left[ \begin{array}{ccc}
0.34 & 0.28 & 0.38
\end{array} 
\right]$
\newline \\ 
The preferences are \\
$ \qquad 34\%  \mbox{ McDonald's} \\
\qquad 28\%  \mbox{ Chipotle} \\
\qquad 38\%  \mbox{ Pizza Hut} \\
$
\part
$ 20\%  \mbox{ McDonald's} \\
\qquad 20\%  \mbox{ Chipotle} \\
\qquad 60\%  \mbox{ Pizza Hut} \\
$

\end{parts}
\end{questions}
\end{document}