\documentclass[addpoints, 12pt]{exam}

\setlength{\topmargin}{-.75in} \addtolength{\textheight}{2.00in}
\setlength{\oddsidemargin}{.00in} \addtolength{\textwidth}{.25in}

\usepackage{amsmath,color,graphicx, multicol}
\usepackage[inline]{asymptote}

\nofiles

\setlength{\parindent}{0in}

\pagestyle{plain}

\title{Calculus exam 2 review}
\author{Wray}
\begin{document}

\begin{asydef}
//
// Global Asymptote definitions can be put here.
//
usepackage("bm");
texpreamble("\def\V#1{\bm{#1}}");
\end{asydef}


\noindent {\sc {\bf {\Large Exam 2 review}}
            \hfill MAT 201, Spring 2017}
\bigskip
\bigskip

\smallskip

In addition to the worksheets discussed on D2L, you should review the following questions.

\bigskip

Use the definition of the derivative to find the derivatives of three kinds of functions.
\begin{questions}
\question
Polynomial: $f(x) = 5 - 2x^2$

\question
Square root: $g(x) = \sqrt{2x + 3}$

\question
Rational: $h(x) = \dfrac{1}{x - 1}$
\end{questions}

\bigskip
Sketch the derivative of each graph shown below.

\begin{questions}
\setcounter{question}{3}
\question
\hspace{1pt}

\begin{asy}
size(7cm);
import graph;
for (int i = -6; i <= 6; ++i)
	{
    draw((-6,i)--(6,i), mediumgray);
    draw((i,-6)--(i,6), mediumgray);
    }
draw(Label("$x$", EndPoint, E), (-7,0)--(7,0), Arrows);
draw(Label("$f(x)$", EndPoint, N), (0,-7)--(0,7), Arrows);
pair F(real t) { 
	return (t, 0.019*(2*t^4 - t^3 - 38*t^2 - 41*t + 30));
}
path g = graph(F, -4.3, 5.4, operator ..);
draw(g);

\end{asy}
\vspace{\stretch{1}}

\question
\hspace{1pt}

\begin{asy}
size(7cm);
import graph;
for (int i = -6; i <= 6; ++i)
	{
    draw((-6,i)--(6,i), mediumgray);
    draw((i,-6)--(i,6), mediumgray);
    }
draw(Label("$x$", EndPoint, E), (-7,0)--(7,0), Arrows);
draw(Label("$f(x)$", EndPoint, N), (0,-7)--(0,7), Arrows);
pair F(real t) { 
	return (t, (1 - 2.0*t)/(t - 1));
}
path g = graph(F, -5.7, 0.87, operator ..);
draw(g);
path h = graph(F, 1.26, 5.8, operator ..);
draw(h);

\end{asy}
\vspace{\stretch{2}}
\clearpage

\question
\hspace{1pt}

\begin{asy}
size(7cm);
import graph;
for (int i = -6; i <= 6; ++i)
	{
    draw((-6,i)--(6,i), mediumgray);
    draw((i,-6)--(i,6), mediumgray);
    }
draw(Label("$x$", EndPoint, E), (-7,0)--(7,0), Arrows);
draw(Label("$f(x)$", EndPoint, N), (0,-7)--(0,7), Arrows);
pair F(real t) { 
	return (t, (2)*t^2 + 4*t - 2.0);
}
pair G(real t) { 
	return (t, -1.0*t + 5);
}
path g = graph(F, -3.2, 1.0, operator ..);
draw(g);
path h = graph(G, 1.0, 5.75, operator ..);
draw(h);

\end{asy}
\vspace{\stretch{2}}

\question
A group of engineers is observing an experimental aircraft.  After $t$ seconds, the height (in feet) of the aircraft above the ground is given by the function
\begin{gather*}
	h(t) = -5t^3 + 30t^2 - 45t.
\end{gather*}

\begin{parts}
\part
How high is the aircraft when the observations begin (at time $t=0$)?

\vspace{\stretch{1}}

\part
Give functions $v(t)$ and $a(t)$ that describe the velocity and acceleration of the aircraft at time $t$.
\vspace{\stretch{1}}

\part
When does the aircraft hit the ground?  What is its velocity when this happens?
\vspace{\stretch{1}}

\part
Find the maximum height of the aircraft.
\vspace{\stretch{1}}

\part
Find the time when the aircraft experiences no acceleration.
\vspace{\stretch{6}}

\end{parts}

\end{questions}
\end{document}