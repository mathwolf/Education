\documentclass[addpoints, 12pt]{exam}

\setlength{\topmargin}{-.75in} \addtolength{\textheight}{2.00in}
\setlength{\oddsidemargin}{.00in} \addtolength{\textwidth}{.25in}

\usepackage{amsmath,color,graphicx, multicol}
\usepackage[inline]{asymptote}

\nofiles

\setlength{\parindent}{0in}

\pagestyle{plain}

\title{Worksheet template}
\author{Wray}
\begin{document}

\begin{asydef}
//
// Global Asymptote definitions can be put here.
//
usepackage("bm");
texpreamble("\def\V#1{\bm{#1}}");
\end{asydef}

\noindent {\sc {\bf {\Large Economic applications of the derivative}}}

\bigskip

\begin{questions}

\question
Suppose that the cost of producing $x$ items is given by the function
\begin{align*}
	C(x) = -0.02x^2 + 50x + 100.
\end{align*}
\begin{parts}
\part
Find functions for the average cost and the marginal cost.
\part
Find the average cost and the marginal cost for the first 100 items.
\part
Find the average cost and the marginal cost for the first 900 items.
\end{parts}

\question
If demand for a product is given by the formula
\begin{align*}
D = f(p),
\end{align*}
then the elasticity of demand is defined to be the relationship
\begin{align*}
E(p) = \dfrac{dD}{dp} \dfrac{p}{D}.
\end{align*}
Suppose that the demand for beef is given by the function
\begin{align*}
D(p) = 386 - 20p.
\end{align*}
Find and graph the elasticity of demand for this product.

\question
If $E < -1$, then demand is said to be elastic.  Otherwise demand is inelastic.  Usinf the model in the last problem, find the prices that make demand for beef inelastic.

\end{questions}

\end{document}