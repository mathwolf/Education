\documentclass[addpoints, 12pt]{exam}

\setlength{\topmargin}{-.75in} \addtolength{\textheight}{2.00in}
\setlength{\oddsidemargin}{.00in} \addtolength{\textwidth}{.25in}

\usepackage{amsmath,color,graphicx, multicol, nicefrac}
\usepackage[inline]{asymptote}

\newcommand{\laplace}[1]{\mathcal{L} \left\lbrace #1 \right\rbrace}

\setlength{\parindent}{0in}

\pagestyle{plain}

\title{Quiz template}
\author{Wray}
\begin{document}

\begin{asydef}
//
// Global Asymptote definitions can be put here.
//
usepackage("bm");
texpreamble("\def\V#1{\bm{#1}}");
\end{asydef}



\noindent {\sc {\bf {\Large Quiz 9}}
            \hfill MAT 201, Spring 2017}
\bigskip

\noindent {\sc  {\large Name:}
             \hfill}
             
\bigskip
\bigskip

\smallskip


\begin{enumerate}

	\item $P(x) = x^{2/3} \left( x^2 - 4 \right)$
	\begin{flushright}
	\begin{asy}
	size(7cm);
	for (int i = -7; i <= 7; ++i)
	{
    	draw((-7,i)--(7,i), mediumgray);
	    draw((i,-7)--(i,7), mediumgray);
    }
	\end{asy}
	\end{flushright}

    \item $Q(x) = x^5 - 4x^3 + 4x - 1$
    
	\begin{flushright}
	\begin{asy}
	size(7cm);
	for (int i = -7; i <= 7; ++i)
	{
    	draw((-7,i)--(7,i), mediumgray);
	    draw((i,-7)--(i,7), mediumgray);
    }
	\end{asy}
	\end{flushright}

\bigskip
\bigskip

\newpage

	\item $R(x) = \dfrac{3x^2 - 12x}{x^2 - 2x - 3}$
    
	\begin{flushright}
	\begin{asy}
	size(7cm);
	for (int i = -7; i <= 7; ++i)
	{
    	draw((-7,i)--(7,i), mediumgray);
	    draw((i,-7)--(i,7), mediumgray);
    }
	\end{asy}
	\end{flushright}

\end{enumerate}

\end{document}