\documentclass[addpoints, 12pt]{exam}

\setlength{\topmargin}{-.75in} \addtolength{\textheight}{2.00in}
\setlength{\oddsidemargin}{.00in} \addtolength{\textwidth}{.25in}

\usepackage{amsmath,color,graphicx, multicol, enumitem}
\usepackage[inline]{asymptote}

\nofiles

\setlength{\parindent}{0in}

\pagestyle{plain}

\title{Quiz template}
\author{Wray}
\begin{document}

\begin{asydef}
//
// Global Asymptote definitions can be put here.
//
usepackage("bm");
texpreamble("\def\V#1{\bm{#1}}");
\end{asydef}



\noindent {\sc {\bf {\Large Quiz 12}}
            \hfill MAT 201, Spring 2017}
\bigskip

\noindent {\sc  {\large Name:}
             \hfill}
             
\bigskip
\bigskip
\begin{questions}

\question
The graph of $f(x)$ is shown in the figure.  Sketch the graph of $f^{\prime}(x)$.

\begin{multicols}{2}

\begin{flushleft}
\begin{asy}
size(6cm);
import graph;
for (int i = -5; i <= 5; ++i)
	{
    draw((-5,i)--(5,i), mediumgray);
    draw((i,-5)--(i,5), mediumgray);
    }
draw(Label("$x$", EndPoint, E), (-6,0)--(6,0), Arrows);
draw(Label("$f(x)$", EndPoint, N), (0,-6)--(0,6), Arrows);
pair F(real t) { 
	return (t, t/(t^2 - t - 6.0) + 1);
	return (t, (-8.0/25.0)*t^2 + (32.0/25.0)*t + 68.0/25.0);
}
path g = graph(F, -4.9, -2.068, operator ..);
draw(g);
path g = graph(F, -1.9, 2.9, operator ..);
draw(g);
path g = graph(F, 3.17, 4.9, operator ..);
draw(g);

\end{asy}
\end{flushleft}

\begin{flushleft}
\begin{asy}
size(6cm);
import graph;
for (int i = -5; i <= 5; ++i)
	{
    draw((-5,i)--(5,i), mediumgray);
    draw((i,-5)--(i,5), mediumgray);
    }
draw(Label("$x$", EndPoint, E), (-6,0)--(6,0), Arrows);
draw(Label("$f^{\prime}(x)$", EndPoint, N), (0,-6)--(0,6), Arrows);
\end{asy}
\end{flushleft}
\end{multicols}

\question
The graph of $f^{\prime}(x)$ is shown in the figure.  Sketch the graph of $f(x)$.

\begin{multicols}{2}

\begin{flushleft}
\begin{asy}
size(6cm);
import graph;
for (int i = -5; i <= 5; ++i)
	{
    draw((-5,i)--(5,i), mediumgray);
    draw((i,-5)--(i,5), mediumgray);
    }
draw(Label("$x$", EndPoint, E), (-6,0)--(6,0), Arrows);
draw(Label("$y$", EndPoint, N), (0,-6)--(0,6), Arrows);
pair F(real t) { 
	return (t, (t + 1.0)*(t - 3.0));
	return (t, (-8.0/25.0)*t^2 + (32.0/25.0)*t + 68.0/25.0);
}
path g = graph(F, -1.95, 3.95, operator ..);
draw(g);
\end{asy}
\end{flushleft}

\begin{flushleft}
\begin{asy}
size(6cm);
import graph;
for (int i = -5; i <= 5; ++i)
	{
    draw((-5,i)--(5,i), mediumgray);
    draw((i,-5)--(i,5), mediumgray);
    }
draw(Label("$x$", EndPoint, E), (-6,0)--(6,0), Arrows);
draw(Label("$y$", EndPoint, N), (0,-6)--(0,6), Arrows);
\end{asy}
\end{flushleft}
\end{multicols}

\question
Out of question 1 and 2 above, which has a second answer?  Sketch your second answer on the  corresponding axes.

\end{questions}

In questions 4 to 6, sketch a graph of a function $f(x)$ that satisfies the given conditions.

\begin{questions}
\setcounter{question}{3}

\question
\begin{multicols}{2}
\raggedcolumns
\begin{gather*}
f^{\prime}(-2) = 0, \> f^{\prime}(0) = 0, \> f^{\prime}(2) = 0 \\
f^{\prime}(x) > 0 \mbox{ if } x \mbox{ is in } (-\infty,-2) \cup (0,2) \\
f^{\prime}(x) < 0 \mbox{ if } x \mbox{ is in } (-2,0) \cup (2,\infty) \\
f^{\prime \prime}(x) > 0 \mbox{ if } x \mbox{ is in } (-1,1) \\
f^{\prime \prime}(x) < 0 \mbox{ if } x \mbox{ is in } (-\infty,-1) \cup (1,\infty)
\end{gather*}
\begin{flushleft}
\begin{asy}
size(6cm);
import graph;
for (int i = -5; i <= 5; ++i)
	{
    draw((-5,i)--(5,i), mediumgray);
    draw((i,-5)--(i,5), mediumgray);
    }
draw(Label("$x$", EndPoint, E), (-6,0)--(6,0), Arrows);
draw(Label("$f(x)$", EndPoint, N), (0,-6)--(0,6), Arrows);
\end{asy}
\end{flushleft}
\end{multicols}

\clearpage

\question
\begin{multicols}{2}
\raggedcolumns
\begin{gather*}
f^{\prime}(-3) = 0 \\
f^{\prime}(x) > 0 \mbox{ if } x \mbox{ is in } (-\infty,-3) \cup (1,\infty) \\
f^{\prime}(x) < 0 \mbox{ if } x \mbox{ is in } (-3,1) \\
\lim_{x \to 1^-} f(x) = -\infty \mbox{ and } \lim_{x \to 1^+} f(x) = -\infty \\
f^{\prime \prime}(x) < 0 \mbox{ if } x \neq 1
\end{gather*}
\begin{flushleft}
\begin{asy}
size(6cm);
import graph;
for (int i = -5; i <= 5; ++i)
	{
    draw((-5,i)--(5,i), mediumgray);
    draw((i,-5)--(i,5), mediumgray);
    }
draw(Label("$x$", EndPoint, E), (-6,0)--(6,0), Arrows);
draw(Label("$f(x)$", EndPoint, N), (0,-6)--(0,6), Arrows);
\end{asy}
\end{flushleft}
\end{multicols}

\question
\begin{multicols}{2}
\raggedcolumns
\begin{gather*}
\mbox{the domain of } f(x) \mbox{ is } (0,\infty) \\
f^{\prime}(x) \mbox{ is always positive} \\
f^{\prime \prime}(x) \mbox{ is always negative}
\end{gather*}
\begin{flushleft}
\begin{asy}
size(6cm);
import graph;
for (int i = -5; i <= 5; ++i)
	{
    draw((-5,i)--(5,i), mediumgray);
    draw((i,-5)--(i,5), mediumgray);
    }
draw(Label("$x$", EndPoint, E), (-6,0)--(6,0), Arrows);
draw(Label("$f(x)$", EndPoint, N), (0,-6)--(0,6), Arrows);
\end{asy}
\end{flushleft}
\end{multicols}

\end{questions}

For the functions in questions 7 to 9:
\begin{enumerate}[label = (\alph*)]
\item Find the domain.
\item Give all the critical points and all the intervals where the function is increasing/decreasing.  Also classify each critical point as a local minimum, local maximum, or neither.
\item Give all inflection points and all the intervals where the function is concave up or concave down.  
\item Sketch the graph.
\end{enumerate}

\begin{questions}
\setcounter{question}{6}

\question
$g(x) = x^4 - 6x^2$

\question
$A(x) = x\sqrt{x + 3}$

\question
$y(x) = (x - 1) e^{-x}$
\end{questions}

\end{document}